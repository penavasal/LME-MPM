\documentclass[preprint,12pt,a4paper]{elsarticle}

\usepackage{lineno}
\usepackage{hyperref}
\usepackage{float}
\usepackage{textcomp}
\usepackage{color}
\usepackage{soul}
\usepackage{cancel}
\usepackage{ulem}
%\usepackage{adjustbox}
\usepackage{rotating}
\usepackage{diagbox}

% Packages to write pseudo-algorithms %
\usepackage{algorithm}
\usepackage{algorithmic}

\usepackage{cancel}

\usepackage{subfigure} % subfiguras

\usepackage{amsmath,amsthm,amssymb}
\usepackage{xstring}


% Tikz
% \usepackage{tikz}
%\usepackage[active,tightpage]{preview}
%\PreviewEnvironment{tikzpicture}
%setlength\PreviewBorder{5pt}
\usepackage{stanli}
\usepackage[ugly]{units}
% \usetikzlibrary{decorations}
% \usetikzlibrary{arrows}
\usetikzlibrary{plotmarks}

\usetikzlibrary{%
    decorations.pathreplacing,%
    decorations.pathmorphing%
  }
\usetikzlibrary{arrows.meta}


\newcommand{\vect}[1]{
  \ensuremath{\mathbf{{#1}}}
}
\newcommand{\tens}[1]{
  \ensuremath{\mathbf{{#1}}}
}
\newcommand{\Matrix}[1]{
  \ensuremath{\mathbf{{#1}}}
}
\newcommand{\Vector}[1]{
  \ensuremath{\mathbf{{#1}}}
}
% Divergence
\newcommand{\Div}[1]{
  \ensuremath{div({#1})}
}
% Gradient
\newcommand\Grad[1]{grad({#1})}
\newcommand\GradS[1]{grad^s({#1})}
\newcommand\GradT[1]{grad^T({#1})}
% Partial derivative
\newcommand{\Deriv}[3][]{
  \ensuremath{\frac{\partial^{#1}{#2}}{ \partial {#3}^{#1} }}
}
% Integral
\newcommand{\Integral}[2]{
  \IfStrEqCase{#1}{
    {2}{\ensuremath{\int_{\Gamma_d}{#2}\ d\Gamma}}
    {3}{\ensuremath{\int_{\Omega}{#2}\ d\Omega}}
  }
}

%%%%%%%%%%%%%%%%%%%%%%%%%%%%%%%%%%%%%%%%%%%%%%%%%%%%%%%%%%%%%%%%%%%%
% Init glossaries
\usepackage[acronym]{glossaries}
\makeglossaries

\newglossaryentry{domain}
{
  name=$\Omega$,
  description={Continuum domain}
}

\newglossaryentry{contour}
{
  name=$\partial\Omega$,
  description={Boundary of the continuum domain $\Omega$. Also
    defined as $\Gamma$}
}

\newglossaryentry{dirichlet-boundary}
{
  name=$ \Gamma_d$,
  description={Essential or Dirichlet boundary conditions over $\partial\Omega$}
}

\newglossaryentry{neumann-boundary}
{
  name=$ \Gamma_n$,
  description={Natural or Neumann boundary conditions over $\partial\Omega$}
}

\newglossaryentry{rho}
{
  name=$\rho$,
  description={Describes the scalar density field}
}

\newglossaryentry{a}
{
  name=$\vec{a}$,
  description={First order tensor which describes the acceleration field}
}

\newglossaryentry{v}
{
  name=$\vec{v}$,
  description={First order tensor which describes the velocity field}
}

\newglossaryentry{u}
{
  name=$\vec{u}$,
  description={First order tensor which describes the displacement field}
}

\newglossaryentry{x}
{
  name=$\vec{x}$,
  description={First order tensor which describes the global coordinates field}
}

\newglossaryentry{xi}
{
  name=$\vec{\xi}$,
  description={First order tensor which describes the local coordinates field}
}


\newglossaryentry{stress}
{
  name=$\tens{\sigma}$,
  description={Second order tensor which means the Cauchy stress tensor}
}


\newglossaryentry{strain}
{
  name=$\tens{\varepsilon}$,
  description={Second order tensor which means the Cauchy strain tensor}
}


\newglossaryentry{Constitutive}
{
  name=$\tens{D}$,
  description={Four order tensor which means the constitutive response
    of the material}
}


\newglossaryentry{LME-beta}
{
  name=$\beta$,
  description={Regularization or thermalization parameter of the
    LME$_{\beta}$ Pareto set}
}

\newglossaryentry{LME-gamma}
{
  name=$\widehat{\gamma}$,
  description={Dimensionless parameter to control the value of the
    thermalization parameter $\beta$ of the \acrshort{lme} shape functions.}
}

\newglossaryentry{gamma-npc}
{
  name=$\gamma$,
  description={Time integration parameter for the \acrshort{npc} algorithm}
}

\newacronym{mpm}{MPM}{Material Point Method}
\newacronym{otm}{OTM}{Optimal Transportation Meshfree}
\newacronym{sph}{SPH}{Smoothed Particle Hydrodynamics}
\newacronym{fem}{FEM}{Finite Element Method}
\newacronym{efgm}{EFGM}{Element-Free Galerkin Method} 
\newacronym{lme}{LME}{Local Maximum-Entropy}
\newacronym{flip}{FLIP}{Fluid Implicit Particle}
\newacronym{pic}{PIC}{Particle in Cell}
\newacronym{gimp}{GIMP}{Generalized Interpolation Material Point}
\newacronym{igimp}{iGIMP}{Implicit GIMP}
\newacronym{ugimp}{uGIMP}{Uniform GIMP}
\newacronym{ddmp}{DDMP}{Dual Domain Material Point}
\newacronym{cpdi}{CPDI}{Convected Particle Domain Interpolation}
\newacronym{ctls}{CTLS}{Conservative Taylor Least Squares}
\newacronym{npc}{NPC}{Newmark Predictor-Corrector}
\newacronym{usl}{USL}{Update Stress Last}
\newacronym{usf}{USF}{Update Stress First}
\newacronym{fe}{FE}{Forward Euler}
\newacronym{dbc}{DBC}{Dirichlet boundary conditions}
\newacronym{nbc}{NBC}{Neumann boundary conditions}

% End glossaries
%%%%%%%%%%%%%%%%%%%%%%%%%%%%%%%%%%%%%%%%%%%%%%%%%%%%%%%%%%%%%%%%%%%%

\newcommand{\CORRECTIONS}[1]{
  \textcolor{green}{{#1}}
}

\newcommand{\MODIFIED}[1]{
  \textcolor{blue}{{#1}}
}


%%%%%%%%%%%%%%%%%%%%%%%%%%%%%%%%%%%%%%%%%%%%%%%%%%%%%%%%%%%%%%%%%%%%

\modulolinenumbers[5]

\journal{Computer Methods in Applied Mechanics and Engineering}
%%%%%%%%%%%%%%%%%%%%%%%
%% Elsevier bibliography styles
%%%%%%%%%%%%%%%%%%%%%%%
%% To change the style, put a % in front of the second line of the current style and
%% remove the % from the second line of the style you would like to use.
%%%%%%%%%%%%%%%%%%%%%%%

%% Numbered
%\bibliographystyle{model1-num-names}

%% Numbered without titles
%\bibliographystyle{model1a-num-names}

%% Harvard
%\bibliographystyle{model2-names.bst}\biboptions{authoryear}

%% Vancouver numbered
%\usepackage{numcompress}\bibliographystyle{model3-num-names}

%% Vancouver name/year
%\usepackage{numcompress}\bibliographystyle{model4-names}\biboptions{authoryear}

%% APA style
%\bibliographystyle{model5-names}\biboptions{authoryear}

%% AMA style
%\usepackage{numcompress}\bibliographystyle{model6-num-names}

%% `Elsevier LaTeX' style
\bibliographystyle{elsarticle-num}
%%%%%%%%%%%%%%%%%%%%%%%

\begin{document}
 

\begin{frontmatter}

\title{On the dynamic assessment of the Local-Maximum Entropy Material Point Method through an Explicit Predictor-Corrector Scheme}

%% Group authors per affiliation:
\author{
Miguel Molinos$^a$,
Pedro Navas$^a$\footnote{Corresponding author: p.navas@upm.es},
Manuel Pastor$^a$
and Miguel Mart\'in Stickle$^a$}
\address{
  $^a$ ETSI Caminos, Canales y Puertos, Universidad Polit\'ectnica de Madrid.\\ c. Prof. Aranguren 3, 28040 Madrid, Spain
}

\begin{abstract}
  \acrfull{mpm} has arisen in the recent years as an alternative to \acrfull{fem} under large
  \textcolor{blue}{deformations}. However, the simulation of shock waves
  propagation in the large deformation regime is still challenging  under this approach due to the incapability of the standard \acrshort{mpm} time
  integration scheme to filter spurious noises. To overcome it, we propose in this paper an explicit Predictor-Corrector time
    integration scheme. Its powerful performance mitigates the
  presence of spurious oscillations with minimal dissipation in high
  frequency problems. Other source of numerical noise in
  \acrshort{mpm} occurs when the material points cross computational grid boundaries, being caused by
  the lack of smoothness of the interpolation functions. This noise
  results in spurious local variations at the material points, where
  strain-stress fields are computed. This could lead to inaccurate
  solutions as well as aborted simulations in the worst cases. To
  surmount it, we propose in this work the \acrfull{lme} approximation
  schemes as a robust substitute of the traditional shape function in
  \acrshort{mpm}. \textcolor{blue}{The performance of both improvements is validated by the high quality
   results of the numerical examples}.
\end{abstract}

\begin{keyword}
  \acrshort{lme} \sep \acrshort{mpm} \sep Explicit predictor-corrector \sep Dynamic problems
\end{keyword}

\section*{}
\centering
\Large
\textbf{On the dynamic assessment of the Local-Maximum Entropy Material Point Method through an Explicit Predictor-Corrector Scheme}\\
\setlength{\parskip}{1cm plus 5mm minus 4mm}
Miguel Molinos, Pedro Navas, Manuel Pastor and Miguel Mart\'in-Stickle.

\setlength{\parskip}{1cm plus 5mm minus 4mm}
%\hline
\setlength{\parskip}{1cm plus 5mm minus 4mm}
\large
\textbf{Highlights}
\normalsize
{\color{blue}
\begin{itemize}
\item An efficient time integration scheme for the \acrshort{mpm} is proposed.
\item The Local Maximum-Entropy approximants as an alternative of the standard \acrshort{mpm} shape functions.
\item This approach might improve large strain fast dynamic MPM simulations. 
\end{itemize}
}
%\begin{highlights}
%\end{highlights}

\end{frontmatter}


\linenumbers

\section{Introduction}
\label{intro}
Since the proposal of \acrshort{mpm} by Sulsky {\it  et al.}
(1994)~\cite{Sulsky1994} as a generalization to solids of the \acrfull{flip} method~\cite{Brackbill1986}, its popularity
has increased due to its ability to deal with large strain regime
without suffering mesh distortion inaccuracies. \textcolor{blue}{One of the main fields where this method is widely used is solid dynamics.}

On one hand, the main source of instability occurs when
material points cross cell boundaries. \textcolor{blue}{This issue led to} the development
of other interpolation techniques to overcome this limitation such as
the \acrfull{gimp} method Bardenhagen \& Kober (2004)~\cite{Bardenhagen2004}, which has
demonstrated to have a good performance in the finite deformation
regime. However, in the absence of a regular grid, construction of the
weighting functions is only achieved at considerable effort and
computational cost.  Furthermore, as it is a voxel-based
discretisation technique, it is prone to suffer voxel domains overlap
or gaps when the material point mesh becomes irregular, which can
introduce severe inaccuracies as noticed by Steffen {\it et al.}
(2008)\cite{Steffen2008}. This is similar to the difficulty
encountered by the finite element method due to element distortion.
A more robust alternative is the \acrfull{ddmp} method proposed by Zhang {\it et al.}
(2011)~\cite{Zhang2011a}. Unfortunately this method shows an
unsatisfactory behaviour when particle/cell ratio
decreases. \textcolor{blue}{Therefore \acrshort{ddmp} requires a large number of particles for convergence~\cite{DHAKAL2016301}, what makes the method computationally expensive.} \textcolor{blue}{To avoid tensile instabilities that are quite common in extension, Sadeghirad {\it et al.} (2011) \cite{Sadeghirad_et_al_IJNME_2011} developed the \acrshort{cpdi} \cite{Sadeghirad_et_al_IJNME_2011}}. In recent years, the employment of spline-lines as shape functions \textcolor{blue}{has gained popularity with the introduction of the B-Spline 
\acrshort{mpm} proposed by Tielen {\it et al.} (2017)~\cite{TIELEN2017265}, in which unstructured set of nodes and
particles can be considered}. More recently, approximants derived from minimization \textcolor{blue}{have
been introduced into the \acrshort{mpm} framework} with the \acrfull{ctls}
reconstruction proposed by Wobbes {\it et al.}
(2018)~\cite{E_Wobbes_2018}. Unfortunately, when particles are spread
in some particular special patterns, the quality of the \acrshort{ctls} approximation decreases locally.

This document adopts the \acrfull{lme}, or Local \textit{Max-Ent} approximates, as a robust substitute of the aforementioned shape functions in \acrshort{mpm}. First introduced by Arroyo \& Ortiz
(2006)~\cite{Arroyo2006}, it belongs to the class of convex 
approximation schemes and provides a seamless transition between
\acrshort{fem} and meshfree interpolations. The \textcolor{red}{\acrshort{lme}}
approximation is based on a compromise between minimizing the
width of the shape function support and maximizing the
entropy of the approximation \cite{Arroyo2006}. The \acrshort{lme} approximation
may be regarded as a regularization, or \textcolor{blue}{(in analogy to statistical mechanics)} \textit{thermalization}, of
Delaunay triangulation which effectively resolves the degenerate cases
resulting from the lack \textcolor{blue}{or uniqueness of the triangulation}. \acrshort{lme} basis functions possess many desirable properties for
meshfree algorithms. First of all, they are entirely defined by the
nodal set and the domain of analysis. They are also non-negative,
satisfy the partition of unity property, and provide an exact
approximation \textcolor{blue}{of affine functions~\cite{Arroyo2006}.} Important contributions on the Maximum-Entropy have been made by Sukumar and coworkers~\cite{Sukumar15} with Cell-based techniques and the ones carried within the \acrfull{otm} method. The latter methodology has been proven to have a good performance under
the dynamic regime being \textcolor{blue}{worth mentioning} the contributions of Li {\it et al.} (2012)~\cite{Li2012} and Navas {\it et al.}
(2018)~\cite{Navas:17b,Navas2018a} in the explicit regime and Navas
{\it et al.}~\cite{Navas2016,Navas2016b,Navas:17c} and Wriggers and
coworkers~\cite{Wriggers18} with implicit schemes. More
recently, under \acrshort{mpm} framework, the work made by Wobbes {\it et al.}(2020)~\cite{Wobbes2020}. The proposed research delves into the benefits of the regularization parameter, \gls{LME-beta}, and the analogy of the different shape functions \textcolor{blue}{derived} by the tuning of this parameter and the traditional \acrshort{mpm} ones.

The aforementioned techniques are devoted to mitigate the
``grid crossing'' error. Nevertheless, in the presence of shock waves, spurious
numerical noises \textcolor{blue}{might} appear despite of the employment of these techniques~\cite{Tran2019e}. These numerical inaccuracies, also known
as wiggles, arise due to inaccuracies in the time discretisation technique.
A simple approach to face those spurious noises lies on the addition of nonphysical damping sources to the equilibrium equations. This
approach has been widely employed in this and many other numerical
techniques. To avoid introducing this nonphysical sources, many
researchers have proposed alternative time integration
schemes which reduce the presence of high frequency noises by
filtering them or increasing somehow the accuracy. One of the first attempts was the proposal of an implicit time integration scheme by Guilkey \& Weiss \cite{Guilkey_2003}. More recently Wang {\it et al.}
\cite{Wang_2016} mitigated these spurious noises by adding a non-viscous
damping to the linear momentum balance equation, and later Charlton
{\it et al.} \cite{Charlton_2017} extended this scheme to the
\acrshort{gimp} approach introducing the \acrfull{igimp}
method. However, the local damping introduced by 
\cite{Wang_2016} can totally over-damp the solution in time-dependent
simulations such as in consolidation process. In a recent publication,
Kularathna \& Soga \cite{Soga_2017} studied an implicit treatment of the pressure in MPM algorithm to simulate material incompressibility
avoiding artificial pressure oscillations by applying
Chorin\textquotesingle s
projection. Within the explicit time integration schemes, Lu {\it et al.}\cite{LU_2018} introduced the time-discontinuous Galerkin method to control the spurious noises
propagation, and later Tran \& Solowski~\cite{Tran2019e}
proposed a generalised-$\alpha$ scheme for \acrshort{mpm} with
promising results but at the expense of increasing the computational
effort. In this paper a less time consuming and high efficient explicit predictor-corrector integration method is
proposed. It consists of an accommodation of the traditional \acrfull{npc} scheme, widely employed in Finite Element methods. \textcolor{blue}{This method
has already been chosen in ~\cite{Navas2018a} among other suitable alternatives} as those proposed
by Wilson {\it et al.} (1972)~\cite{Wilson1972} or Chung \& Hulbert
(1993)~\cite{Geranlized_alpha_1993} because its simplicity and good
performance dealing with solid dynamic problems under a meshfree
framework.  
\textcolor{blue}{Other well known source of energy dissipation and numerical noise is the presence of a non-trivial null space of the linear operator that maps particles values onto nodal values. It was earlier identified by Brackbill (1984)~\cite{BRACKBILL1988469} as ringing instability in the \acrfull{pic} method. A recent development in the field of particle methods is a null-space filter engineered by Gritton \&
Berzins (2017)~\cite{Gritton2017} which overcome this limitation but still introduce unwelcome damping.  Later, Hammerquist \& Nairn (2017)~\cite{HAMMERQUIST2017724} introduced the XPIC as a parametric extension of the aforementioned research with excellent results.}

The aim of this document is to mitigate the spurious oscillations due to
inaccuracies in both space and time discretisation by the employment of a
suitable combination of the \acrshort{lme} family shape functions, and the proposal of a predictor-corrector scheme. The advantages of
this approach will be illustrated through several demanding test cases on the elastic regime: the
propagation of shock waves in an elastic bar and the response of a block of soil gradually loaded with gravitational forces.

The article is organised as follows : Section \ref{sec:meshfree-methodology}
is devoted to describe the meshfree methodology adopted in this
research, first \acrshort{mpm} procedure is introduced in
\ref{sec:derivation-mpm}, second the predictor-corrector
time integration scheme is presented in \ref{sec:epc-algor-mpm}, and
third \acrshort{lme} approximation scheme will be introduced in
\ref{sec:local-max-ent}. In Section
\ref{sec:Application-linear-elasticity-dynamic-problems}, applications to prove the numerical accuracy of the proposed approach are
presented. Finally, conclusions and future research topics are exposed in Section \ref{sec:conclusions}.


\section{The meshfree methodology}
\label{sec:meshfree-methodology}

The aim of this section is to provide an overview of the standard
explicit \acrshort{mpm} algorithm~\cite{Sulsky1994}. Without loosing
generality, the method consists of three main steps: (i) a
variational recovery process, where particle data is projected onto the
grid nodes, (ii) an Eulerian step, where balance of momentum equation
is expressed as a nodal equilibrium equation through a \acrshort{fem}-like
procedure, and finally (iii) a Lagrangian advection of the
particles. In consequence, \acrshort{mpm} can be regarded as a
Lagrangian-Eulerian method where particles carry on all the physical
information and a set of background nodes is employed to
compute the equilibrium equation. \textcolor{blue}{The described methodology is sketched by the scheme of the figure~\ref{fig:MPM_algorithm}.}
%%%%%%%%%%%%%%%%%%%%%%%%%%%%%%%%%%%%%%%%%%%
\begin{figure}
%\sidecaption
  \centering
  \resizebox{0.8\hsize}{!}{
    \includegraphics[width=\textwidth]{./Figures-MPM-scheme-horizontal}
  }
  \caption{Description of the three steps in \acrshort{mpm} standard algorithm.}
  \label{fig:MPM_algorithm}
\end{figure}
%%%%%%%%%%%%%%%%%%%%%%%%%%%%%%%%%%%%%%%%%%
 In what follows, we will adopt the following convention: \textcolor{blue}{Any vector field is defined with an overarrow as $\vec{\square}$}. Three kind of subscripts or superscripts are  considered:  subscript $\square_p$ is used to define a particle variable,  subscript $\square_I$ is reserved for nodal variables, superscript $\square^{\psi}$ involves a virtual magnitude. \textcolor{blue}{The convention adopted for the operators is as folows:} $\dot{\square}$ and $\ddot{\square}$ \textcolor{blue}{are considered} for the first and second time derivative, $\square \otimes \square$ means the dyadic operator, $\square \cdot \square$ and $\square \colon \square$ \textcolor{blue}{are the single and double contraction of tensor index}, $\Div{\square}$ denotes the divergence operator, and finally $\Grad{\square}$ and $\GradS{\square}$ denotes the gradient and its symmetric part. Following, the \acrshort{mpm} methodology, the explicit predictor-corrector scheme and \acrshort{lme} approximation shape functions are \textcolor{red}{described} in subsection \ref{sec:derivation-mpm}, \ref{sec:epc-algor-mpm} and \ref{sec:local-max-ent} respectively.

\subsection{Derivation of MPM procedure}
\label{sec:derivation-mpm}

In \acrshort{mpm} the continuum mechanics approach is considered. We consider a region \gls{domain} occupied by an elastic body like
the sketched in the figure~\ref{fig:Continuum-solid}, and \gls{contour} the boundaries of the domain defined by $\partial \Omega
= \Gamma_d \cup \Gamma_n$  \textcolor{red}{where} $\Gamma_d \cap  \Gamma_n = \emptyset$.
%%%%%%%%%%%%%%%%%%%%%%%%%%%%%%%%%%%%%%%%%%%%%%%%%%%%%%%%%%%%%%%%%%%%%%%%%%
\begin{figure}
%\sidecaption
  \centering
  \resizebox{0.5\hsize}{!}{
    \includegraphics[width=\textwidth]{Figures-Continuum-solid}}
  \caption{Description of the boundary-value-problem in a
    continuum. Red lines represents the closure $\partial \Omega$
    of the domain $\Omega$ represented in gray.}
  \label{fig:Continuum-solid}
\end{figure}
%%%%%%%%%%%%%%%%%%%%%%%%%%%%%%%%%%%%%%%%%%%%%%%%%%%%%%%%%%%%%%%%%%%%%%%%%%
In this context the field \gls{u} allows to describe the \textit{global state}
of the system. Now the variable $\phi =
(\tens{\varepsilon},\tens{\sigma})$ is defined as the set of \textit{local
  states} at any point of the continuum which can be derived from the
field \gls{u} through the following set of governing equations and
restrictions that must be satisfied. First (i) we relate global to local state by the \textit{compatibility equation} \textcolor{blue}{where the strain field \gls{strain} is defined by}:
\begin{equation}
  \label{eq:Compatibility-equation}
  \tens{\varepsilon} = \GradS{\vec{u}},
\end{equation}
\textcolor{blue}{together with essential or \acrfull{dbc} at \gls{dirichlet-boundary}.} We will further assume that 
strains are infinitesimal. The stress field \gls{stress} is considered the corresponding
conjugate variable for the strain field, being the one which
satisfies (ii) the \textit{conservation of \textcolor{red}{linear} momentum equation}:
\begin{equation}
  \label{eq:Balance-momentum}
\rho \frac{D\vec{v}}{Dt} = \Div{\tens{\sigma}} + \rho \vec{b}
\end{equation}
\textcolor{blue}{together with the natural of \acrfull{nbc} at \gls{neumann-boundary}. An additional component is (iii) the constitutive equation which relates stress and strain increments as,}
{\color{red}
\begin{equation}
  \label{eq:Constitutive-equation}
\tens{\Delta \sigma} = \tens{D} \colon \tens{\Delta \varepsilon}.
\end{equation}
}
\textcolor{blue}{
In this research, plane strain linear elasticity is considered. The final equation of the set is (iv) the mass conservation, which can be expressed by} 
\begin{equation}
  \label{eq:Rho-material-derivative}
    \dot{\rho} + \rho \Div{\vec{v}} = 0.
\end{equation}

In order to obtain the variational statement of the problem, let us define a
virtual displacement field such that
\begin{equation}
  \label{eq:Hilbert-space}
  \vec{u}^{\psi} \in \mathcal{H}^1_0(\Omega) = \{ \vec{u}^{\psi} \in
  \mathcal{H}^1 \mid \vec{u}^{\psi} = \vec{0}\ \text{on}\ \Gamma_d \};
\end{equation}
\textcolor{blue}{where $\vec{u}^{\psi} \in \mathcal{H}^1$ means that}
\begin{equation}
  \label{eq:cauchy-secuence}
  \Integral{3}{\vec{u}^{\psi}} < \infty\ \quad\text{and}\quad
  \Integral{3}{\tens{\varepsilon}^{\psi}} < \infty.
\end{equation}
The principle of virtual work states that the equilibrium solution to
the boundary-value problem of elasticity is the function $\vec{u} \in
\mathcal{H}^1_0$ such that, for any $\vec{u}^{\psi} \in
\mathcal{H}^1_0$,
the following holds:
%%%%%%%%%%%%%%%%%%%%%
\begin{equation}
  \label{eq:BalanceMomentum_wf}
  \Integral{3}{\rho\ \left( \frac{d\vec{v}}{dt}\ - \vec{b} \right) \cdot \vec{u}^{\psi}} =
  \Integral{2}{\vec{t}\ \cdot \vec{u}^{\psi}} - \Integral{3}{\tens{\sigma} \colon
   \tens{\varepsilon}^{\psi}}.
\end{equation}
%%%%%%%%%%%%%%%%%%%%%
Thus, equation~\eqref{eq:BalanceMomentum_wf}, together with
\eqref{eq:Constitutive-equation} and
\eqref{eq:Rho-material-derivative}, represents the weak form of the problem. Next, we will discretise the set of equations of the mathematical model using a double discretisation procedure, see figure~\ref{fig:MPM-discretization}.\\

\textcolor{blue}{First, the velocity field and the virtual displacements fields are discretised in a finite set of nodes $\textbf{X} = \{ \vec{x}_I, I \in \mathcal{B} \} \subset \mathbb{R}^{2}$, where $\mathcal{B} = 1, \ldots, n_I$. An approximation of the continuum field can be obtained with the help of nodal values and appropriate interpolation functions $N_I(\vec{x})$. Also spatial derivatives of those quantities, such as gradients and divergences, are computed through the support of the background set of nodes as}
\begin{align}
    \label{eq:variable_reconstruction}
    \varphi(\vec{x}) &= \sum_{I\ \in\ \mathcal{B}} N_I(\vec{x})\ \varphi_I \\
    \label{eq:grad_variable_reconstruction}
    \Grad{\varphi}(\vec{x}) &= \sum_{I\ \in\ \mathcal{B}} \varphi_I\ \otimes \Grad{N_I(\vec{x})}
\end{align}
\textcolor{blue}{Secondly, the continuum \gls{domain} is discretised with a finite set of material points (also known as particles in this manuscript) $\hat{\Omega} = \{ \vec{x}_p, p \in \mathcal{C} \} \subset \Omega$, where $\mathcal{C} = 1, \ldots, n_p$. Any particle field such as position, velocity, mass, volume and stress denoted by $\vec{x}_p$, $\vec{v}_p$, $m_p$, $V_p$ and $\tens{\sigma}_p$, respectively, are assigned to each material point. Furthermore, any particle field $\varphi_p$ can be approximated within the nodes in the neighbourhood of each particle, $\mathcal{B}_p \subset \mathcal{B}$, and evaluating the interpolation function in the position of each particle as}
\begin{align}
    \label{eq:particle_variable_reconstruction}
\varphi_p &= \sum_{I\ \in\ \mathcal{B}_p} N_I(\vec{x}_p)\ \varphi_I \\
\Grad{\varphi_p} &= \sum_{I\ \in\ \mathcal{B}_p} \varphi_I \otimes \Grad{N_I(\vec{x}_p)}
\end{align}
\textcolor{blue}{The integrals appearing in the weak form \eqref{eq:BalanceMomentum_wf} are evaluated by means of the Riemann integral \cite{Riemann_1854} applied to a finite set of points with associated volumes $V_p$ interpreted as quadrature weights.}
\begin{equation}
    \label{eq:particle_riemann_integral}
\Integral{3}{\varphi} = \sum_{p\ \in\ \mathcal{C}} \varphi_p\ V_p 
\end{equation}
%%%%%%%%%%%%%%%%%%%%%%%%%%%%%%%%%%%%%%%%%%%%%%%%%%%%%%%%%%%%%%%%%%%%%%%%%%
\begin{figure}
%\sidecaption
  \centering
  \resizebox{0.5\hsize}{!}{
    \includegraphics[width=\textwidth]{Figures-Mesh-particles-back}}
  \caption{Description of the spatial discretisation for domain presented in the
    figure~\ref{fig:Continuum-solid}. Blue mesh represents the
    background computational support, and the red mesh conforms the
    discretised continuum body.}
  \label{fig:MPM-discretization}
\end{figure}
%%%%%%%%%%%%%%%%%%%%%%%%%%%%%%%%%%%%%%%%%%%%%%%%%%%%%%%%%%%%%%%%%%%%%%%%%%
\textcolor{blue}{Let us illustrate the procedure described above by fully developing the term of the acceleration forces in \eqref{eq:BalanceMomentum_wf}. Employing the definition \eqref{eq:variable_reconstruction}, the velocity and virtual displacement fields can be approximated within its nodal values. Hence, the acceleration forces is reduced to}
\begin{equation}
    \label{eq:particle_acceleration_forces_I}
    \sum_{I,J\ \in\ \mathcal{B}} \Integral{3}{ \rho\ N_J(\vec{x})  \frac{d\vec{v}_J}{dt}\ \cdot N_I(\vec{x})\vec{u}_I^{\psi}}.
\end{equation}
\textcolor{blue}{Rearranging terms in \ref{eq:particle_acceleration_forces_I} and keeping in mind that $\vec{u}_I^{\psi}$ and $\vec{v}_J$ are nodal evaluations of $\vec{u}^{\psi}$ and $\vec{v}$, respectively,  the following expression is obtained}
\begin{equation}
    \label{eq:particle_acceleration_forces_II}
    \sum_{I,J\ \in\ \mathcal{B}} \vec{u}_I^{\psi} \cdot \Integral{3}{ N_I(\vec{x})\ \rho\ N_J(\vec{x})}\ \frac{d\vec{v}_J}{dt}.
\end{equation}
\textcolor{blue}{and by means of \eqref{eq:particle_riemann_integral}, and keeping in mind that expression \eqref{eq:particle_acceleration_forces_II} is for any $\vec{u}^{\psi} \in
\mathcal{H}^1_0$, the following discrete expression of the acceleration can be derived}
%%%%%%%%%%%%%%%%
\begin{equation}
\label{eq:particle_acceleration_forces_III}
\begin{split}
&\sum_{I,J\ \in\ \mathcal{B}}\ \sum_{p\ \in\ \mathcal{C}} N_{Ip}\ \rho_p\ N_{Jp}\ V_p\ \frac{d\vec{v}_J}{dt} = \\
&=\sum_{I,J\ \in\ \mathcal{B}}\ \sum_{p\ \in\ \mathcal{C}} N_{Ip}\ m_p\ N_{Jp}\ \frac{d\vec{v}_J}{dt} = \\
&= \tens{m}_{IJ}\ \frac{d\vec{v}_J}{dt}
\end{split}
\end{equation}
%%%%%%%%%%%%%%%%
\textcolor{blue}{Where $\tens{m}_{IJ}$, is the nodal mass matrix, and $N_{Ip}, N_{Jp}$ are the interpolation functions for nodes $I,J$ evaluated at the position of each particle $p$. In order to improve computational efficiency and stability, the nodal mass matrix can be substituted by the lumped mass matrix $\tens{m}_{IJ}^{lumped}$. The remaining terms of \eqref{eq:BalanceMomentum_wf} are obtained within a similar procedure and can be found in literature, see for instance \cite{Zhang_book_2016}. Therefore equation \eqref{eq:BalanceMomentum_wf} can be discretised as}
%%%%%%%%%%%%%%%%%%%%%
\begin{equation}
\label{eq:BalanceMomentum_wf_discretized}
    \tens{m}_{IJ}\ \frac{d\vec{v}_J}{dt} = \sum_{p\ \in\ \mathcal{C}} - \underbrace{\tens{\sigma}_{p} \cdot \Grad{N_{Ip}} V_p}_{f^{int}_I}\ + \underbrace{N_{Ip}\ \vec{b}_{p}\ m_p  +\ N_{Ip}\ \vec{t}_{p}\ V_p\ h^{-1}}_{f^{ext}_I}
\end{equation}
%%%%%%%%%%%%%%%%%%%%%
\textcolor{blue}{Where $h$ is the solid thickness in 2D and $\tens{\sigma}_{p} = \tens{\sigma}_{p}(\tens{\varepsilon}_{p})$
is the stress field at particle $p$, which can be obtained after considering a suitable constitutive model. The particle strain field is approximated thorough the time integration of the rate of strain tensor, that is computed employing the velocity at the background set of nodes by means of the equation}
%%%%%%%%%%%%%%%%%%%%%
\begin{equation}
  \label{eq:IncrStrainPoint}
  \dot{\tens{\varepsilon}_{p}} = \frac{\Delta
    \tens{\varepsilon}_{p}}{\Delta t} = \sum_{I\ \in\ \mathcal{B}_p}
  \frac{1}{2} \left[\Grad{N_{Ip}}\ \otimes \vec{v}_{I} + \vec{v}_{I} \otimes
    \Grad{N_{Ip}}\ \right].
\end{equation}
%%%%%%%%%%%%%%%%%%%%%
Finally, mass conservation is guaranteed by \textcolor{blue}{means of \eqref{eq:Rho-material-derivative} which can be rewritten in terms of the rate of strain tensor $\dot{\tens{\varepsilon}}$ as}
\begin{equation}
  \label{eq:MassConservation}
\dot{\rho} = - \rho\ \mathit{trace} \left( \dot{\tens{\varepsilon}} \right).
\end{equation}
\textcolor{blue}{Equation \eqref{eq:BalanceMomentum_wf_discretized} is a second order ordinary differential equation in time and a time integration scheme is required.} \textcolor{blue}{To this end, time is discretised into a finite set of time steps $k = 1\ldots ,Nt$, where $k$ is the current time step and $N_t$ is the total number of time steps.} Once the nodal equilibrium equation is solved, the values at the nodes are interpolated back into the particles, which are advected
to the new position through:
\begin{equation}
  \label{eq:Updated_Lagrangian}
  \dot{\vec{v}}_p = \sum_{I\ \in\ \mathcal{B}_p} N_{Ip}\ \vec{a}_{I},\quad \text{and} \quad
  \dot{\vec{x}}_{p} = \sum_{I\ \in\ \mathcal{B}_p} N_{Ip}\ \vec{v}_{I}.  
\end{equation}
Traditionally, Eqs. \eqref{eq:BalanceMomentum_wf_discretized} and ~\eqref{eq:Updated_Lagrangian},
are solved with an explicit forward Euler algorithm. In the following subsection, \textcolor{blue}{this well known algorithm and the proposed schemes are described.}

\subsection{MPM time integration scheme: the Newmark Predictor-Corrector proposal}
%\subsection{Explicit predictor-corrector scheme for \acrshort{mpm}.}
\label{sec:epc-algor-mpm}

As stated previously, an explicit forward Euler algorithm has been utilized widely within the \acrshort{mpm} methodology. This scheme has been described in detail by many researchers
\cite{Sulsky1994}, \cite{Bardenhagen2002}, \cite{thesis_Andersen_2009}. Other authors have proposed many others time integration alternatives like \cite{Guilkey_2003,Charlton_2017,Tran2019e}. In the
first publication on \acrshort{mpm} \cite{Sulsky1994}, \textcolor{blue}{the nodal acceleration
was considered to update the particles by}
\begin{equation}
  \label{eq:Sulsky-1994-UL-v}
  \vec{v}_p^{k+1} = \vec{v}_p^{k} + \sum_{I\ \in\ \mathcal{B}_p} \Delta t\ N_{Ip}\ \vec{a}_{I}^{k},
\end{equation}
\begin{equation}
  \label{eq:Sulsky-1994-UL-x}
  \vec{x}_p^{k+1} = \vec{x}_p^{k} + \sum_{I\ \in\ \mathcal{B}_p} \Delta t\ N_{Ip}\ \vec{v}_{I}^{k}.
\end{equation}
However, as Andersen (2009)\cite{thesis_Andersen_2009} pointed out, this algorithm has been shown to be numerically unstable due to \textcolor{red}{the fact} that nodal forces can be infinite for an infinitesimal nodal mass $\tens{m}_I$. This issue may lead to numerical problems when nodal acceleration is obtained in the evaluation of the Eqs. \eqref{eq:Sulsky-1994-UL-x} and \eqref{eq:Sulsky-1994-UL-v}. Hence, a
corrected version of this algorithm was proposed by Zhang {\it et al.}
(2016)\cite{Zhang_book_2016}:
\begin{equation}
  \label{eq:Zhang-2016-UL-x}
  \vec{x}_p^{k+1} = \vec{x}_p^{k} + \sum_{I\ \in\ \mathcal{B}_p} \Delta t\ \frac{N_{Ip}\ \vec{p}_{I}^{k}}{\tens{m}_I}, 
\end{equation}
\begin{equation}
  \label{eq:Zhang-2016-UL-v}
  \vec{v}_p^{k+1} = \vec{v}_p^{k} + \sum_{I\ \in\ \mathcal{B}_p} \Delta t\ \frac{N_{Ip}\ \vec{f}_{I}^{k}}{\tens{m}_I}.
\end{equation}
Delving into the improvement of the accuracy of the \acrshort{mpm} explicit schemes, Tran \& Solowski (2019)\cite{Tran2019e} presented a
generalized-$\alpha$ scheme for \acrshort{mpm} inspired in the explicit time
integration algorithm proposed by Chung \& Hulbert
(1993)\cite{Geranlized_alpha_1993}, but with the particularity that
the acceleration is evaluated both \textcolor{red}{at} the beginning and \textcolor{red}{at} the end of the time step.
\begin{equation}
  \label{eq:Tran-2019-GA-v}
  \vec{v}_p^{k+1} = \vec{v}_p^{k} + \sum_{I\ \in\ \mathcal{B}_p} \Delta t\  N_{Ip}\ \left[(1 - \gamma)\ \vec{a}_I^{k} +
    \gamma\ \vec{a}_I^{k+1} \right],\\
\end{equation}
\begin{equation}
\label{eq:Tran-2019-GA-x}
  \vec{x}_p^{k+1} = \vec{x}_p^{k} + \sum_{I\ \in\ \mathcal{B}_p} N_{Ip} \left[ \Delta t\ \vec{v}_{I}^{k}+ \Delta t^2\left( (\frac{1}{2} - \beta)\
    \vec{a}_{I}^{k} + \beta\ \vec{a}_{I}^{k+1} \right) \right],
\end{equation}
\begin{equation}
  \label{eq:Tran-2019-GA-a}
  \vec{a}_p^{k+1} = \sum_{I\ \in\ \mathcal{B}_p} N_{Ip}\ \vec{a}_{I}^{k+1}.
\end{equation}

This scheme has been proven to damp out the highest frequency noises
\cite{Tran2019e}. However, it can present the same numerical instabilities
as in \eqref{eq:Sulsky-1994-UL-x},\eqref{eq:Sulsky-1994-UL-v} when
nodal masses become infinitesimal, and requires extra storage for
nodal values of acceleration and previous steps.

\textcolor{blue}{In this section, an explicit predictor-corrector time integration
scheme is proposed, which is based on the Newmark central differences explicit scheme.} This method is devoted to solve  system of equations of the type
{\color{red}
\begin{equation*}
 \sum_{J\ \in\ \mathcal{B}} \Matrix{M}_{IJ}\ddot{\Vector{d}}_{J} + \Matrix{C}_{IJ}\dot{\Vector{d}}_{J} +
  \Matrix{K}_{IJ}\Vector{d}_{J} = \Vector{F}_{I}.
\end{equation*}
}
\textcolor{blue}{The presence of a background grid of nodes} allows to apply this \textcolor{red}{integration} method within the \acrshort{mpm} framework in a similar manner that the one
proposed by Tran~\textit{et al.}~\cite{Tran2019e}. By using the predictor \textcolor{blue}{step}  it is possible to calculate nodal velocities and update particles position employing nodal values
of velocity and acceleration. 

The predictor-corrector algorithm has
been described in the classic literature \cite{Hughes2000}, and its
stability and computational advantages were widely validated by Liu
\cite{Xiaojian94}. The ``classic'' \acrfull{npc} algorithm starts with a
predicted value of the nodal velocities at the $(k+1)$th time step, {\color{blue}denoted by $\vec{v}_{I}^{pred}$, which is calculated as follows:
\begin{equation}
  \label{eq:Predictor-velocity-I}
  \vec{v}_{I}^{pred} = \vec{v}_I^k + (1 - \gamma)\ \Delta t\ \vec{a}_I^k.
\end{equation}
}
The \textit{user-defined}
parameter $\gamma \geq 0$ that appears in \eqref{eq:Predictor-velocity-I}, influences both the predictor accuracy
and the stability of the algorithm. As pointed out Liu
\cite{Xiaojian94}, the truncation error of the predictor formula is
$O(\Delta t^3)$ when $\gamma = 0.5$, and is unconditionally stable if
$ 0 < \gamma \leq 0.25$. To accommodate this step to \acrshort{mpm} framework, it is necessary to get the nodal values of the velocity and acceleration throughout a variational recovery process where particles quantities are transferred to the mesh nodes. This technique arises as a generalization of the super-convergent recovery
procedures described by Zienkiewicz \& Zhu \cite{ZZ1992_I} (\textit{ZZ}) in the context of \acrshort{fem}. Bardenhagen \& Kober \cite{Bardenhagen2004} proved that through this information-transference technique mass and momentum are conserved. For a general particle variable $\Phi_p$ it is possible to get its nodal homologous $\Phi_I$ \textcolor{blue}{by means of the \textit{ZZ} technique as:}
{\color{red}
\begin{equation}
  \label{eq:Variational-recovery}
   \Phi_I = \sum_{p\ \in\ \mathcal{C}} \frac{m_p N_{Ip} \Phi_p}{m_I}.
 \end{equation}
 }
 Therefore, to get an analogous expression for
 \eqref{eq:Predictor-velocity-I} in the context of \acrshort{mpm}, the
 procedure described in the equation \eqref{eq:Variational-recovery}
 is employed, \textcolor{red}{obtaining} the following expression:
 {\color{red}
 \begin{equation}
   \label{eq:Predictor-velocity-II}
   \vec{v}_{I}^{pred} = \sum_{p\ \in\ \mathcal{C}} \underbrace{\frac{N_{Ip} m_p
       \vec{v}_p^k}{m_I}}_{\vec{v}_I^{k}} + (1 - \gamma)\ \Delta t\  \underbrace{\frac{N_{Ip} m_p \vec{a}_p^k}{m_I}}_{\vec{a}_I^{k}}.
 \end{equation}
 }
\textcolor{blue}{However}, this way of computing the predictor stage can introduce instabilities due to numerical cancellation likewise the original Sulky algorithm. This \textcolor{blue}{issue} can be avoided easily by the equivalent formulation inspired in the \textit{ZZ} technique,
{\color{red}
\begin{equation}
  \label{eq:Predictor-velocity-III}
  \vec{v}_{I}^{pred} =  \sum_{p\ \in\ \mathcal{C}} \frac{ N_{Ip} m_p (\vec{v}_p^k + (1 - \gamma)\ \Delta t\ \vec{a}_p^k)}{m_I}.
\end{equation}
}
This way of computing the nodal predictor is numerically stable
and minimize the computational effort. Once nodal velocities are
obtained, the \acrshort{dbc} are imposed over \gls{dirichlet-boundary}. \textcolor{blue}{Next the \textit{corrector}  step is introduced. Predicted nodal velocities obtained in \eqref{eq:Predictor-velocity-III}, are now corrected by the equation}
\begin{equation}
  \label{eq:Corrector-velocity}
  \vec{v}_{I}^{k+1} = \vec{v}_{I}^{pred} + \gamma\ \Delta t\ \frac{\vec{f}_{I}^{k}}{\tens{m}_I}.
\end{equation}
Finally particle accelerations, velocities and positions are updated as,
\begin{align}
      \label{eq:Update-lagrangian-pce-a}    
        &\vec{a}_p^{k+1} = \sum_{I\ \in\ \mathcal{B}_p} \frac{N_{Ip}\vec{f}_{I}^{k}}{\tens{m}_I}\\
        \label{eq:Update-lagrangian-pce-v}  
      &\vec{v}_p^{k+1} = \vec{v}_p^n + \sum_{I\ \in\ \mathcal{B}_p} \Delta t\
        \frac{N_{Ip}\
        \vec{f}_{I}^{k}}{\tens{m}_I}\\
        \label{eq:Update-lagrangian-pce-u}
      &\vec{x}_p^{k+1} = \vec{x}_p^n + \sum_{I\ \in\ \mathcal{B}_p} \Delta t\
         N_{Ip}\ \vec{v}_{I}^{k} +
        \frac{1}{2}\Delta t^2\ \frac{N_{Ip}\
        \vec{f}_{I}^{k}}{\tens{m}_I}.
\end{align}
\textcolor{blue}{In the proposed scheme, accelerations \eqref{eq:Update-lagrangian-pce-a} and velocities \eqref{eq:Update-lagrangian-pce-v} are computed with information coming from the predictor step. On the other hand, positions \eqref{eq:Update-lagrangian-pce-u} are evaluated with information coming from the corrector step in the velocity term, while acceleration term introduces information from the corrector. Therefore, it can be appreciated some similarities with the \textit{leapfrog scheme} \cite{Zhang_book_2016},} where position is not updated at full time step, but the velocity is updated at half time steps. Notice also that, with this approach, the calculation of nodal momentum values are not required. Due to its simplicity, the proposed scheme can be implemented with minor modifications over the standard forward Euler. The full implementation is summarized in the algorithm \ref{alg:the_npc_alg}.

\begin{algorithm}
\caption{\acrfull{npc} scheme}
\label{alg:the_npc_alg}
  \begin{algorithmic}[1]
    %%%%%%%%%%%%%%%%%%%%%%%%%%%%%%%%%%%%%%%%%%%%%%%%%%%%%%%%%%%%%%%%%%%%%%%%%%%%%%%%%%%%%% º
    \STATE \textbf{Update mass matrix}.
    %%%%%%%%%%%%%%%%%%%%%%%%%%%%%%%%%%%%%%%%%%%%%%%%%%%%%%%%%%%%%%%%%%%%%%%%%%%%%%%%%%%%%% 
    \STATE \textbf{Explicit Newmark Predictor}.\\
    \begin{equation*}
      \vec{v}_I^{pred} = \sum_{p\ \in\ \mathcal{C}} \frac{ N_{Ip}^{k} m_p (\vec{v}_p^k + (1 - \gamma)\ \Delta t\ \vec{a}_p^k)}{m_I}.
    \end{equation*}
    %%%%%%%%%%%%%%%%%%%%%%%%%%%%%%%%%%%%%%%%%%%%%%%%%%%%%%%%%%%%%%%%%%%%%%%%%%%%%%%%%%%%%% 
    \STATE \textbf{Impose essential boundary conditions}.\\
    At the fixed boundary, set $\vec{v}_{I}^{pred} = 0$. 
    %%%%%%%%%%%%%%%%%%%%%%%%%%%%%%%%%%%%%%%%%%%%%%%%%%%%%%%%%%%%%%%%%%%%%%%%%%%%%%%%%%%%%% 
    % \STATE \textbf{Discard the previous nodal values}.
    %%%%%%%%%%%%%%%%%%%%%%%%%%%%%%%%%%%%%%%%%%%%%%%%%%%%%%%%%%%%%%%%%%%%%%%%%%%%%%%%%%%%%% 
    \STATE \textbf{Deformation tensor increment calculation}.
    \begin{equation*}
      \dot{\tens{\varepsilon}_{p}}^{k+1} = \sum_{I\ \in\ \mathcal{B}_p} \left[ \vec{v}_{I}^{pred} \otimes \Grad{N_{Ip}^{k+1}} \right]^s\ \quad \text{and}\ \quad \Delta \tens{\varepsilon}_{p}^{k+1} = \Delta t\ \dot{\tens{\varepsilon}_{p}}^{k+1}.
    \end{equation*}
    %%%%%%%%%%%%%%%%%%%%%%%%%%%%%%%%%%%%%%%%%%%%%%%%%%%%%%%%%%%%%%%%%%%%%%%%%%%%%%%%%%%%%% 
    \STATE \textbf{Update the density field}.
    \begin{equation*}
      \rho_p^{k+1} = \frac{\rho_p^k}{1 + \mathit{trace}\left[\Delta\tens{\varepsilon}_{p}^{k+1}\right]}.
    \end{equation*}
    %%%%%%%%%%%%%%%%%%%%%%%%%%%%%%%%%%%%%%%%%%%%%%%%%%%%%%%%%%%%%%%%%%%%%%%%%%%%%%%%%%%%%% 
    \STATE \textbf{Balance of forces calculation}.\\
    Calculate the total grid nodal forces by evaluating the right hand side of 
    \eqref{eq:BalanceMomentum_wf_discretized} with the information from the predictor step.
    In those nodes where $\Deriv{\vec{v}_I^{k}}{t} \big\rvert_{\Gamma_d} = 0$, the acceleration is fixed to zero and nodal forces are stored as reactions.\\
    %%%%%%%%%%%%%%%%%%%%%%%%%%%%%%%%%%%%%%%%%%%%%%%%%%%%%%%%%%%%%%%%%%%%%%%%%%%%%%%%%%%%%% 
    \STATE \textbf{Explicit Newmark Corrector}.
    \begin{equation*}
      \vec{v}_{I}^{k+1} = \vec{v}_{I}^{pred} + \gamma\ \Delta t\ \frac{\vec{f}_{I}^{k+1}}{\tens{m}_I^{k+1}}.
    \end{equation*}
    %%%%%%%%%%%%%%%%%%%%%%%%%%%%%%%%%%%%%%%%%%%%%%%%%%%%%%%%%%%%%%%%%%%%%%%%%%%%%%%%%%%%%%
    \STATE \textbf{Update particles lagrangian quantities}.
    \begin{align*}
      &\vec{a}_p^{k+1} = \sum_{I\ \in\ \mathcal{B}_p} \frac{N_{Ip}^k\vec{f}_{I}^{k}}{\tens{m}_I^k},\\
      &\vec{v}_p^{k+1} = \vec{v}_p^n + \sum_{I\ \in\ \mathcal{B}_p} \Delta t\
        \frac{N_{Ip}^k\
        \vec{f}_{I}^{k}}{\tens{m}_I^k},\\
      &\vec{x}_p^{k+1} = \vec{x}_p^n + \sum_{I\ \in\ \mathcal{B}_p} \Delta t\
         N_{Ip}^k\ \vec{v}_{I}^{k} +
        \frac{1}{2}\Delta t^2\ \frac{N_{Ip}^k\
        \vec{f}_{I}^{k}}{\tens{m}_I^k}.
    \end{align*}
    %%%%%%%%%%%%%%%%%%%%%%%%%%%%%%%%%%%%%%%%%%%%%%%%%%%%%%%%%%%%%%%%%%%%%%%%%%%%%%%%%%%%%% 
    \STATE \textbf{Reset nodal values}.
  \end{algorithmic}
\end{algorithm}

\subsection{Local \textit{Max-Ent} approximants}
\label{sec:local-max-ent}
The popularity of the \acrshort{mpm} has increased notoriously during
the recent years due to its ability to deal with large strain problems without mesh distorsion issues inherent to mesh based methods like \acrshort{fem}, see Wi{\c{e}}ckowski \cite{Wieckowski2004}. However, in the simulations
made with the original \acrshort{mpm}, numerical noises appear when particles cross the cell boundaries. Solving this issue is the main goal of the employment of the \acrshort{lme} shape functions.
\acrfull{lme} approximation schemes were
first introduced by Arroyo \& Ortiz (2006)\cite{Arroyo2006} and has been recently tested under \acrshort{mpm} framework by Wobbes {\it et al.} (2020)\cite{Wobbes2020}. The simulations presented in \cite{Wobbes2020} of \acrshort{mpm} within \acrshort{lme} show considerably more accurate stress approximations than traditional \acrshort{mpm} schemes. However, how the regularization parameter $\beta$ affects to the accuracy and stability of the solution is not assessed deeply in that research. The tuning of this $\beta$ parameter allows to make the comparison of the accuracy against analogous traditional \acrshort{mpm} shape function.

The basic idea of the shape functions based on such an estimate is to interpret the shape function $N_I(\vec{x})$ as a probability. This allows us to introduce two important limits:
the principle of maximum-entropy (\textit{max-ent}) statistical
inference stated by \cite{Jaynes1957}, and the Delaunay triangulation
which ensures the minimal width of the shape function. 

This approximation scheme represents an optimal compromise, in the sense of Pareto, between the \textit{unbiased statistical inference} based on
the nodal data which leads to the principle of \textit{Maximum-Entropy}
stated by Jaynes \cite{Jaynes1957}, and the definition of local shape
functions of \textit{least width} the least biased shape functions.

\textcolor{blue}{Following \cite{Arroyo2006}, entropy of a discrete random variable can be defined as the uncertainty of the random variable. A measure of this uncertainty can be obtained by means of Shannon's entropy}:
\begin{equation}
  \label{eq:Shannon-entropy}
  H(p_1,\ldots,p_n) = -\sum^{N_n}_{I=1}{p_I\ \log p_I }
\end{equation}
where $p_I$ \textcolor{blue}{stands for probability of the random variable outcomes. By interpreting these probabilities as the shape functions $N_I(\vec{x})$ of an approximation scheme, equation \eqref{eq:Shannon-entropy} can be regarded as a measure of the uncertainty of the approximation. Thus, according to Jaynes's principle of maximum entropy \cite{Jaynes1957},} the least-biased approximation scheme  \textcolor{red}{can be} given by
\begin{align*}
  \label{eq:least-biased-approximation-scheme}
  \text{(LME)} \hspace{0.15cm} &\text{Maximize} \hspace{0.15cm} H(N_I) \equiv
  -\sum_{I}^{N_n}{N_I(\vec{x})\log N_I }\\
  &\text{subject to}\
  \begin{cases}
    N_I \ge 0, \hspace{0.15cm} \text{I=1, ..., n} \\[1em]   
    \sum\limits_{I=1}^{N_n}{N_I} = 1 \\[1em]   
    \sum\limits_{I=1}^{N_n}{N_I \vec{x}_I} = \vec{x} \\
  \end{cases}
\end{align*}
On the other hand, the control of the shape function width and its
decay with distance away from the corresponding nodes is a desirable property. To reach to this objective \cite{Arroyo2006} propose the following linear program,
\begin{align*}
  %\label{eq:RAJAN}
  \text{(RAJ)} \hspace{0.15cm} &\text{For fixed} \hspace{0.15cm}
  \vec{x} \hspace{0.15cm} \text{minimize} \hspace{0.15cm} U(\vec{x}_p,N_I) \equiv
\sum_I N_I |\vec{x}_p - \vec{x}_I |^2\\
  &\text{subject to}\
  \begin{cases}
    N_I \ge 0, \hspace{0.15cm} \text{I=1, ..., n} \\[1em]   
    \sum\limits_{I=1}^{N_n}{N_I} = 1 \\[1em]   
    \sum\limits_{I=1}^{N_n}{N_I \vec{x}_I} = \vec{x} \\
  \end{cases}
\end{align*}
\textcolor{blue}{To reach a compromise between the two competing objectives, a Pareto set is considered}  \cite{Arroyo2006}
\begin{align*}
  %\label{eq:LME-scheme-pareto-set}
  \text{(LME)}_{\beta} \hspace{0.15cm} &\text{For fixed} \hspace{0.15cm}
  \vec{x} \hspace{0.15cm} \text{minimize} \hspace{0.15cm} f_{\beta}(\vec{x}, N_I) \equiv \beta U(\vec{x},N_I) - H(N_I) \\
  &\text{subject to}\
  \begin{cases}
    N_I \ge 0, \hspace{0.15cm} \text{I=1, ..., n} \\[1em]   
    \sum\limits_{I=1}^{N_n}{N_I} = 1 \\[1em]   
    \sum\limits_{I=1}^{N_n}{N_I \vec{x}_I} = \vec{x} \\
  \end{cases}
\end{align*}
The regularization or \textit{thermalization} parameter
between the two criterion, $\beta$, has Pareto optimal values in the range
$\beta \in (0,\infty)$. The unique solution of
the local \textit{max-ent} problem \acrshort{lme}$_\beta$ is:
\begin{equation}
  \label{eq:LME-p}
N_I^*(\vec{x})=\frac{\exp\left[ -\beta \; |\vec{x}-\vec{x}_I|^2 +
    \vec{\lambda}^* \cdot (\vec{x}-\vec{x}_I) \right] } {Z(\vec{x},\vec{\lambda}^*)}
\end{equation}
where
\begin{equation}
  \label{eq:LME-Z}
Z(\vec{x}, {\vec{\lambda}}) = \sum_{I=1}^{N_n}{ \exp \left[ -\beta \; |\vec{x}-\vec{x}_I|^2 + \vec{\lambda} \cdot (\vec{x}-\vec{x}_I)  \right]}
\end{equation}
being $\vec{\lambda}^*(\vec{x})$ the unique minimiser for the function $\log
Z(\vec{x}, \vec{\lambda})$. The traditional way to obtain such a minimiser is using Eq.~(\ref{eq:LME-J}) to calculate small increments of $\partial\vec{\lambda}$ in a Newton-Raphson approach. $\tens{J}$ is defined as the Hessian matrix, obtained by:
\begin{eqnarray}
  \label{eq:LME-J} 
  \tens{J}(\vec{x}, \vec{\lambda},\beta) &\equiv& \frac{\partial
                                                  \vec{r}}{\partial \vec{\lambda}}\\
  \label{eq:LME-r}
  \vec{r}(\vec{x},\vec{\lambda},\beta) &\equiv& \frac{\partial \log{ Z(   \vec{x},\vec{\lambda}})}{\partial \vec{\lambda}}  = \sum_I^{N_n} p_I(\vec{x},\vec{\lambda},\beta) \, (\vec{x} - \vec{x}_I)
\end{eqnarray}
In order to obtain the first derivatives of the shape function, it is also necessary to compute~$\nabla N_I^*$
\begin{equation}
  \label{eq:LME-grad-p}
\nabla N_I^* = N^*_I  \, \left(\nabla f^*_I-\sum_J^{N_n} N^*_J \, \nabla f^*_J\right)
\end{equation}
where
\begin{equation}
  \label{eq:LME-f}
f^*_I(\vec{x},  \vec{\lambda},\beta)=-\beta \, |\vec{x}-\vec{x}_I|^2 + \vec{\lambda}   \,  (\vec{x}-\vec{x}_I)
\end{equation}
Employing the chain rule, rearranging and considering $\beta$ as a constant, Arroyo and Ortiz~\cite{Arroyo2006} obtained the following expression for the gradient of the shape function.
\begin{eqnarray}
\nabla N_I^* &=& -N_I^* \,  (\tens{J}^*)^{-1} \,  (\vec{x} - \vec{x}_I) \label{eq26} 
\end{eqnarray}
The regularization parameter $\beta$ of \acrshort{lme} shape functions may be controlled by adjusting a dimensionless parameter\footnote{\textcolor{blue}{To avoid confusion with the \gls{gamma-npc} parameter of the \acrshort{npc}, the dimensionless parameter defined by Arroyo \& Ortiz \cite{Arroyo2006} as $\gamma$ will be represent by \gls{LME-gamma} to preserve as much as possible the original notation.}}, \textcolor{blue}{$\widehat{\gamma}=\beta h^2$} \cite{Arroyo2006}, where $h$ is defined as a measure of the nodal
spacing. 
Since $N_I$ is defined in the entire domain, in practice, the
function $\exp(-\beta \vec{r} )$ truncated  by  a given tolerance, 10$^{-6}$ in this research,  would ensure a reasonable range of neighbours (see \cite{Arroyo2006} for details).
This tolerance defines the limit values of the influence radius and is used thereafter to find the neighbour nodes of a given integration point. An additional remark is that, analogous to alternative non-polynomial meshfree basis functions, the \acrshort{lme} approximation scheme requires more than $d+1$ nodes to determine the values of the shape functions as well as their derivatives at any point in the convex hull of the nodal set, where $d$ is the dimension of the problem.

This interpolation technique avoids important shortcomings when using
\acrshort{gimp} or B-Spline \acrshort{mpm} regarding the computational domain boundaries (see Steffen {\it et al.} (2008)\cite{Steffen2008b}), which are related to additional considerations in the application of the boundary
conditions. Motivated by their increased extents, particles may share an influence radius that lies outside of the simulation domain. Some researchers have solved this problem with the so called ``extra'' or ``ghost'' nodes. These nodes require especial treatment, similar to those employed in the \acrfull{sph}, for
further details see Liu \& Liu (2003)\cite{Liu2003}. The approach here described does not require the employment of this artifices.
Due to the \acrshort{fem}-compatibility, the \acrshort{lme} shape
function is degenerated to linear finite element shape function if $d+1$ neighbouring nodes are chosen as the support, where $d$ is the number of dimensions in the problem. \textcolor{blue}{Figure \ref{fig:LME_MPM} shows that larger values of the parameter $\beta$ make the solution to tend to the linear \acrshort{fem} solution as the athermal limit is reached \cite{Arroyo2006}}. Ullah {\it et al.} \cite{Augarde_2013} took advantage of the \acrshort{lme} \acrshort{fem}-compatibility to couple the \acrfull{efgm} and \acrshort{fem} for linear elasticity and for problems with both material and geometrical non-linearities. Furthermore, with a conveniently adopted \textit{regularization} parameter it is possible to get a \acrshort{gimp}-like shape function. Finally \acrshort{sph}-like behaviour can be obtained for lower values of \textcolor{blue}{$\widehat{\gamma}$} since the support of the shape function is drastically increased, and therefore \textit{smoother} solutions are obtained. See \cite{Navas2016} for an application of this capability, where oscillations due to excess of pore water pressure in consolidation problems are smoothed out by using this technique. The employment of smoothing algorithms is also straightforward in the fluid-solid interaction problems \cite{Arduino_2018}. \textcolor{blue}{This behaviour was noticed previously by \cite{Arroyo2006}, were authors highlighted how, by adjusting the spatial variation of $\beta(\vec{x})$, it is possible to select regions of the domain of analysis which are treated by finite elements and regions that are treated in the style of meshfree methods, with seamless transitions between those regions. The aforementioned adaptability can be appreciated in figure~\ref{fig:LME_MPM}.} \\
%%%%%%%%%%%%%%%%%%%%%%%%%%%%%%%%%%%%%%%%%%%%%%%%%%%%%%%%%%%%%%%%%%%%%%%%%%
\begin{figure*}
  \centering
  \subfigure[Q4]{    
    \begin{tabular}{c}
      \includegraphics[width=0.14\textwidth]{Figures-MPM-Shape-Fun}\\
      \includegraphics[width=0.14\textwidth]{Figures-MPM-Shape-Fun-dx}\\
      \includegraphics[width=0.14\textwidth]{Figures-MPM-Shape-Fun-dy}
    \end{tabular}
  }
  \subfigure[$\text{LME}_{17}$]{
    \begin{tabular}{c}      
      \includegraphics[width=0.14\textwidth]{Figures-LME-17-3-Shape-Fun}\\
      \includegraphics[width=0.14\textwidth]{Figures-LME-17-3-Shape-Fun-dx}\\
      \includegraphics[width=0.14\textwidth]{Figures-LME-17-3-Shape-Fun-dy}
    \end{tabular}
  }
  \subfigure[uGIMP]{
    \begin{tabular}{c}
      \includegraphics[width=0.14\textwidth]{Figures-GIMP-Shape-Fun}\\
      \includegraphics[width=0.14\textwidth]{Figures-GIMP-Shape-Fun-dx}\\
      \includegraphics[width=0.14\textwidth]{Figures-GIMP-Shape-Fun-dy}
    \end{tabular}
  }
  \subfigure[$\text{LME}_{10}$]{
    \begin{tabular}{c}
     \includegraphics[width=0.14\textwidth]{Figures-LME-10-0-Shape-Fun}\\ 		   \includegraphics[width=0.14\textwidth]{Figures-LME-10-0-Shape-Fun-dx}\\      \includegraphics[width=0.14\textwidth]{Figures-LME-10-0-Shape-Fun-dy}    
    \end{tabular}
  }
  \subfigure[$\text{LME}_{5}$]{
    \begin{tabular}{c}
      \includegraphics[width=0.14\textwidth]{Figures-LME-5-0-Shape-Fun}\\
      \includegraphics[width=0.14\textwidth]{Figures-LME-5-0-Shape-Fun-dx}\\
      \includegraphics[width=0.14\textwidth]{Figures-LME-5-0-Shape-Fun-dy}
    \end{tabular}
  }
  \caption{Comparative of linear piecewise shape functions (Q4) and
    \acrshort{ugimp} shape functions \textit{versus}  \acrshort{lme}
    approximation for a two-dimensional arrangement of nodes, and
    spatial derivatives for several values of $\widehat{\gamma} = \beta h^2$.}
  \label{fig:LME_MPM}
\end{figure*}
%%%%%%%%%%%%%%%%%%%%%%%%%%%%%%%%%%%%%%%%%%%%%%%%%%%%%%%%%%%%%%%%%%%%%%%%%% 
In this research and in \cite{Arroyo2006}, $\beta$ is a scalar as the
influence area of the shape function is controlled by the Euclidean
norm, therefore the search area is geometrically a circle in 2D, or a
sphere in 3D. Building upon the idea of anisotropic shape functions,
\cite{Kochmann2019} introduced an enhanced version of the original
\acrshort{lme} scheme, which uses an anisotropic support to deal with 
tensile instability. This is another benefit of the proposed methodology, that, although is out of the scope of the present document, will be incorporated in future research.

\textcolor{blue}{Unfortunately, the employment of \textit{max-ent} approximants is not free of pitfalls. Some requirements such as the need for a convex domain, as well as those derived from its calculation (the determination of the Lagrange Multipliers, specially at the boundaries, or the obtainment of the minimizer of the logarithmic function), make the usage of it a challenging tool. Concerning the convex domain requirement, Arroyo \& Ortiz~\cite{Arroyo2006} briefly discussed the existence of non-convex domains, and proposed some solutions to it. For instance, the possibility of replacing the Euclidean distance $\lVert x - x_a \rVert$ in the definition of the shape functions by the length of the shortest path contained within the domain connecting $x$ and $x_a$; or by decomposing the non-convex domain into convex sub-domains. Of course the requirement of convex domain is not exclusive from \textit{max-ent}, it also concerns to the remaining interpolation techniques in the \acrshort{mpm} and other meshfree methods. For instance, this topic has been extensively studied in the context of MLS-based meshfree methods, for instance visibility, diffraction, and constrained path criteria. These methods are directly applicable to local \textit{max-ent} approximation. }

\textcolor{blue}{Despite these drawbacks, the results depicted in this manuscript will strengthen the motivation of its employment in order to mitigate typical \acrshort{mpm} problems such as cell-crossing or stress instabilities in a wide range of problems.}

\section{Application to linear elasticity dynamic problems.}
\label{sec:Application-linear-elasticity-dynamic-problems}

This section is devoted to test the ability of both predictor-corrector
time integration scheme and the Local \textit{Max-Ent} approximants to
overcome spurious oscillations due to the grid crossing and high
frequency loads under the context of \acrshort{mpm}. Three different tests have been adopted for this purpose: The first example is the well known benchmark proposed by Dyka \& Ingel (1995)\cite{Dyka1995}. It is devoted to test the accuracy of the \acrfull{npc} scheme. The second example is the test proposed in the PhD thesis of Andersen (2009)\cite{thesis_Andersen_2009} where the evolution of velocity waves in an elastic square. In the third example the present approach is tested under a scenario where shocks and grid crossing occurs. All simulations were performed with in-house software.

\subsection{Dyka\textquotesingle s bar \cite{Dyka1995}}
\label{sec:dyka-bar}

This benchmark was proposed by Dyka~\cite{Dyka1995} since allows to study easily the capability
of the proposed time integration algorithm to avoid velocity
field instabilities. It consists of a one-dimensional bar of a length
of 0.1333 meters, sketched in the figure~\ref{fig:Dyka_Bar}. The
boundary conditions are: displacements are constrained ($\vec{v}
\rvert_{x=L} = 0$) in the right border, being free on all other boundaries. An initial velocity of $\vec{v}_o = - 5\ m/s$ is given to the
left quarter of the bar. Finally, the elastic parameters chosen for this test are:
\begin{itemize} 
\item  Density : $7833\ kg/m3$
\item  Poisson ratio : $0$
\item  Elastic modulus : $200 \cdot 10^9\ Pa$
\end{itemize}
%%%%%%%%%%%%%%%%%%%%%%%%%%%%%%%%%%%%%%%%%%%%%%%%%%%%%%%%%%%%%%%%%%%%%%%%%%
\begin{figure}
%\sidecaption
  \centering
  \resizebox{\hsize}{!}{
    \begin{tikzpicture} 
  \scaling{2}; 
  % Nodos 
  \point{a}{0}{1};
  \point{b}{1.25}{1};
  \point{c}{5}{1};
  % Barras
  \beam{2}{a}{b};
  \beam{2}{b}{c};
  % Apoyos
  \support{3}{c}[90];
  % Fuerzas
  \lineload{4}{b}{a}[1][0.2];
  \notation{5}{a}{b}[$-5\ m/s$][.5][below][2];
  % Nombres de nodos
  \notation{1}{a}{A}[below left];
  \notation{1}{b}{B}[below left];
  \notation{1}{c}{C}[below left];
  % Cotas
  \dimensioning{1}{a}{b}{0.5}[{\unit[1/4]{L}}];
  \dimensioning{1}{b}{c}{0.5}[{\unit[3/4]{L}}];
\end{tikzpicture}
}
  \caption{Geometrical description of the Dyka \cite{Dyka1995} bar.}
  \label{fig:Dyka_Bar}
\end{figure}
%%%%%%%%%%%%%%%%%%%%%%%%%%%%%%%%%%%%%%%%%%%%%%%%%%%%%%%%%%%%%%%%%%%%%%%%%%
In the proposed example, the simulation extends from time zero up to \textcolor{red}{$10^{-4}$} s.
This time interval allows the elastic wave to travel a distance of 2.6
lengths of the bar. For the spatial discretisation, a set of seven nodal mesh sizes (0.1, 0.3325, 0.5, 1.0, 3.3325, 6.665, 10.0 millimeters) are considered. For each element a number of four particles was selected. In the initial layout, particles are located the exact quadrature points of a linear quadrilateral, with the exception of the \acrshort{ugimp} simulation, where gaps or overlap between
voxels of each particle are not allowed. In those cases, each particle occupies the center of each cell quarter. For all simulations, time step is controlled by a Courant-Friedrichs-Levy condition of 0.1 , were the adopted celerity is computed as:
\begin{equation}
  \label{eq:Cel}
  Cel = \max\{\max_{p \in \Omega_p}\{ \vec{v}_p \} , \max_{p \in \Omega_p}\{ \sqrt{\frac{E_p}{\rho_p}} \} \}.
\end{equation}
An important consideration regarding modellization concerns the
background mesh. Notice that free border of the bar has a maximum
horizontal displacement of 0.03 millimeters, therefore 
a computational domain with an extra gap of 0.03 millimeters is
required in order to accommodate the unconstrained displacement of the
particles in the left border of the bar. This problem arises
when the mesh size is small enough that relative displacement of the particles is larger \textcolor{red}{than} the distance to the border, so grid crossing phenomena could appear even in those cases with infinitesimal displacements. In this case, an analytical solution can be obtained through the characteristics method, described in the appendix~\ref{app:analytical_sol}. \textcolor{blue}{This section contains a large number of comparisons, which have been summarised in Table \ref{tab:dyka-cases-summarized}. First, a convergence study regarding the integration scheme and interpolation technique is performed in  \ref{sec:Convergence-analysis}. Second, comparison of the capabilities of the \acrshort{npc} and \acrshort{fe} algorithms to damp out numerical noise is analysed in \ref{sec:FE-vs-NPC}. Third, the response of the \acrshort{lme} approximation for different $\widehat{\gamma}$ values is study in  \ref{sec:Sensitive-analysis-gamma}. Fourth, comparison between the well-known \acrshort{ugimp} shape function against the \acrshort{lme} approximation is performed in \ref{sec:LME-vs-uGIMP}. Finally, a comparison between the \acrshort{mpm} and the \acrfull{otm} method, using the same time integration scheme and interpolation technique, is conducted in \ref{sec:OTM-vs-MPM}.}

%%%%%%%%%%%%%%%%%%%%%%%%%%%%%%%%%%%%%%%%%%%%%%%%%%%%%%%%%%%%%%%%%%%%%%%%%% 
\begin{sidewaystable}
\centering
\begin{tabular}{c|c|c|c|c|c|}
\cline{2-6}
                                    & \ref{sec:Convergence-analysis} 
                                    & \ref{sec:FE-vs-NPC} 
                                    & \ref{sec:Sensitive-analysis-gamma} 
                                    & \ref{sec:LME-vs-uGIMP} & 
                                    \ref{sec:OTM-vs-MPM}\\ \hline
\multicolumn{1}{|l|}{\diagbox[]{MPM-Q4-FE }{MPM-Q4-NPC }}  & 
\diagbox[]{x}{x} &  
\diagbox[]{x}{x} & 
 & 
 & 
 \\ \hline
\multicolumn{1}{|l|}{\diagbox[]{MPM-uGIMP-FE}{MPM-uGIMP-NPC}} &  \diagbox[]{x}{x} & 
 &
 &
\diagbox[]{ -}{x} &
 \\ \hline
\multicolumn{1}{|l|}{\diagbox[]{MPM-LME$_{2.0}$-FE}{MPM-LME$_{2.0}$-NPC}} & \diagbox[]{x}{x} &  
 &
 \diagbox[]{ -}{x} &
 &    
 \\ \hline
\multicolumn{1}{|l|}{\diagbox[]{MPM-LME$_{3.0}$-FE}{MPM-LME$_{3.0}$-NPC}} & \diagbox[]{x}{x} & 
 &
 \diagbox[]{ -}{x}  &
 &
 \\ \hline
\multicolumn{1}{|l|}{\diagbox[]{MPM-LME$_{4.0}$-FE}{MPM-LME$_{4.0}$-NPC}} & \diagbox[]{x}{x} &
 & 
 \diagbox[]{ -}{x} &
 \diagbox[]{ -}{x} &
 \diagbox[]{ -}{x} \\ \hline
\multicolumn{1}{|l|}{\diagbox[]{OTM-LME$_{4.0}$-FE}{OTM-LME$_{4.0}$-NPC}} &
 &
 & 
 &
 &
 \diagbox[]{ -}{x}  \\ \hline
\end{tabular}
\caption{Table with all the permutations for the Dyka bar \cite{Dyka1995}.}
\label{tab:dyka-cases-summarized}
\end{sidewaystable}
%%%%%%%%%%%%%%%%%%%%%%%%%%%%%%%%%%%%%%%%%%%%%%%%%%%%%%%%%%%%%%%%%%%%%%%%%% 

\subsubsection{Convergence analysis}
\label{sec:Convergence-analysis}

To measure the convergence of the solutions
for the different time integration and approximation schemes the
root-mean-square (RMS) error in the velocity field is computed. RMS
error is defined as
\begin{equation}
  \label{eq:RMS}
  RMS = \sqrt{\frac{1}{N} \sum^{N}_p \left( \vec{v}_p - \hat{\vec{v}}_p \right)^2},
\end{equation}
where $\vec{v}_p$ and $\hat{\vec{v}}_p$ are respectively the analytical and
numerical solutions evaluated in the final time step in the position
of each particle. In Fig. \ref{fig:Dyka-error-evol} the evolution of the RMS is obtained for both time integration schemes. 
%%%%%%%%%%%%%%%%%%%%%%%%%%%%%%%%%%%%%%%%%%%%%%%%%%%%%%%%%%%%%%%%%%%%%%%%%% 
\begin{figure*}
%\sidecaption
  \centering
  \resizebox{\hsize}{!}{
    \includegraphics[width=\textwidth]{./Figures-Error-evol}
  }
  \caption{Velocity error evolution at the point A in the Dyka's bar ,
    convergence plots for \acrshort{fe} and \acrshort{npc}. The plot is subdivided with
    colours, the darker part of the diagram shows coincides when the
    relative movement of the particles is large enough to produce the
    grid crossing phenomena. The lightest part of the diagram
    coincides when the relative movement of the particles in
    negligible in comparison with the mesh size. And in the middle
    region a transition behaviour take place.}
  \label{fig:Dyka-error-evol}
\end{figure*}
%%%%%%%%%%%%%%%%%%%%%%%%%%%%%%%%%%%%%%%%%%%%%%%%%%%%%%%%%%%%%%%%%%%%%%%%%%
The right figure, with the \acrshort{npc} results, shows lower values of the estimated error, denoting the higher performance of this methodology. About the spatial discretisation, the \acrshort{lme} schemes show an error comparable to the obtained with the \acrshort{ugimp}, being even lower close to the \textit{grid-crossing region}. \textcolor{blue}{In this region the performance is punished with significant movement of the particles as far as the mesh size is reduced.}

\textcolor{blue}{Figure \eqref{fig:Dyka-error-evol} also highlights how on the one hand the slope of the \acrshort{lme} error decreases monotonously for the lower $\widehat{\gamma}$ values up to the \textit{grid-crossing region} (dark grey region), where more error is accumulated. On the other hand, when higher values of $\widehat{\gamma}$ are considered in \acrshort{lme} and also for the \acrshort{ugimp} and the bi-linear shape functions, the change in error slope occurs in the \textit{transition region} (middle grey region). It might be concluded that the latter methods are more sensitive to grid crossing than the former}.

\textcolor{blue}{Finally,  the absence of \acrshort{ugimp} values for a mesh size of 0.3325 and 0.5 millimeters in figure \eqref{fig:Dyka-error-evol} is remarkable. The reason is due to an unstable increasing error suffered during \acrshort{ugimp} simulations for these mesh sizes. A feasible explanation for this phenomena could be the presence of numerical cancellation which could produce gaps between voxels. Further research should be done in this direction for getting a better comprehension of this phenomenon. Conversely, this shortcoming is not suffered by \acrshort{lme}, independently of regular or irregular nodal layout.}

\subsubsection{FE \textit{versus} NPC}
\label{sec:FE-vs-NPC}

\textcolor{blue}{A comparison of the performance of both time integration schemes is presented in figure \ref{fig:Dyka-NPC-FE}. Both integration schemes show numerical oscillations with respect to the analytical solution all over the simulation time, being worse just after jump discontinuities. However, the \acrshort{npc} approach damps out these oscillations faster than the \acrshort{fe} scheme. The oscillations of the \acrshort{npc} almost disappeared before reaching subsequent jump discontinuities, while those of \acrshort{fe} do not. Moreover, oscillations of the \acrshort{fe} seems to increase further after each jump discontinuity and an unstable tendency can by foreseen. This unstable tendency is not appreciated in the \acrshort{npc} integration scheme.} 
%%%%%%%%%%%%%%%%%%%%%%%%%%%%%%%%%%%%%%%%%%%%%%%%%%%%%%%%%%%%%%%%%%%%%%%%%%
\begin{figure}
%\sidecaption
  \centering
  \resizebox{0.9\hsize}{!}{
    \includegraphics[width=\textwidth]{./Figures-Velocity-FE-vs-PCE-CFL-01}
  }
  \caption{Comparison of \acrshort{npc} and \acrshort{fe}
      performances: In the picture the velocity evolution at the point in the bar left side
    is plotted.}
  \label{fig:Dyka-NPC-FE}
\end{figure}
%%%%%%%%%%%%%%%%%%%%%%%%%%%%%%%%%%%%%%%%%%%%%%%%%%%%%%%%%%%%%%%%%%%%%%%%%%

\subsubsection{Sensitive analysis to \texorpdfstring{$\widehat{\gamma}$}{gamma}  parameter}
\label{sec:Sensitive-analysis-gamma}

\textcolor{blue}{Figure \eqref{fig:Dyka-LME-gamma} shows the sensitivity of the \acrshort{lme}
approximation scheme to variations in the parameter
$\widehat{\gamma}$, which is the one that controls the value of the regularization parameter
$\beta$ together with the nodal spacing parameter $h$. It can be observed that for lower values of $\widehat{\gamma}$ the numerical solution presents a faster decay of the spurious oscillations. This capability could be useful in simulations where
extremely noisy oscillations could damage the solutions like memory materials.  An additional observation, concerning to the
solution sensibility depending on the regularization parameters, is the behaviour of the solution depending on the decreasing of mesh size. For larger mesh sizes, where the relative particle displacement is negligible in comparison with the cell size, the global behaviour is \acrshort{fem}-like, therefore, larger values of $\widehat{\gamma}$ may offer better results. On the other hand, when mesh size is small enough to produce grid-crossing, meshfree behaviour is required to ensure the convergence of the solution and tiny values of $\widehat{\gamma}$ may lead to better performances.}
%%%%%%%%%%%%%%%%%%%%%%%%%%%%%%%%%%%%%%%%%%%%%%%%%%%%%%%%%%%%%%%%%%%%%%%%%%
\begin{figure}
%\sidecaption
  \centering
  \resizebox{0.9\hsize}{!}{
    \includegraphics[width=\textwidth]{./Figures-Velocity-LME-gamma-comparative}
  }
  \caption{Sensitivity  of LME approximants performance to changes in the
    dimensionless regularization parameter \textcolor{blue}{$\widehat{\gamma}$}. To illustrate it, the velocity evolution at the point in the bar left side is plotted.}
  \label{fig:Dyka-LME-gamma}
\end{figure}
%%%%%%%%%%%%%%%%%%%%%%%%%%%%%%%%%%%%%%%%%%%%%%%%%%%%%%%%%%%%%%%%%%%%%%%%%%

\subsubsection{LME \textit{versus} uGIMP}
\label{sec:LME-vs-uGIMP}

\textcolor{blue}{The performance of the \acrshort{ugimp}~\cite{Bardenhagen2004} shape function \textit{versus} the \acrshort{lme} approximation scheme with a dimensionless regularization parameter $\widehat{\gamma}$ of 4.0 is compared in Figure \ref{fig:Dyka-uGIMP-LME}.
%%%%%%%%%%%%%%%%%%%%%%%%%%%%%%%%%%%%%%%%%%%%%%%%%%%%%%
\begin{figure}
%\sidecaption
  \centering
  \resizebox{0.9\hsize}{!}{
    \includegraphics[width=\textwidth]{./Figures-Velocity-uGIMP-vs-LME-Dyka}
  }
  \caption{Velocity evolution at the point in the bar left side.}
  \label{fig:Dyka-uGIMP-LME}
\end{figure}
%%%%%%%%%%%%%%%%%%%%%%%%%%%%%%%%%%%%%%%%%%%%%%%%%%%%%%%
 Although remarkable differences are not observed under a regular mesh, \acrshort{lme} approximants seems to fits better than \acrshort{ugimp} after the jump.}

\subsubsection{OTM \textit{versus} MPM}
\label{sec:OTM-vs-MPM}

\textcolor{blue}{Finally, the \acrshort{mpm} approach is compared with the \acrfull{otm}~\cite{Li2010} method, both with the same time integration scheme, spatial discretisation and interpolation technique. Results of the comparison are shown in figure \ref{fig:Dyka-OTM-MPM}. During the first half of the simulation both methods seem to perform in a similar way, but during the second half of the simulation after the elastic wave has traveled from the free border to
the fixed one and back, in \acrshort{otm} the solution becomes more noisy than the one performed by \acrshort{mpm}.}   
%%%%%%%%%%%%%%%%%%%%%%%%%%%%%%%%%%%%%%%%%%%%%%%%%%%%%%%%%%%%%%%%%%%%%%%%%%
\begin{figure}
%\sidecaption
  \centering
  \resizebox{0.9\hsize}{!}{
    \includegraphics[width=\textwidth]{./Figures-Velocity-MPM-vs-OTM-Dyka}
  }
  \caption{Velocity evolution at the point in the bar left side.}
  \label{fig:Dyka-OTM-MPM}
\end{figure}
%%%%%%%%%%%%%%%%%%%%%%%%%%%%%%%%%%%%%%%%%%%%%%%%%%%%%%%%%%%%%%%%%%%%%%%%%%

\subsection{Rigid block}
\label{sec:andersen-block}

The following test was proposed to validate the ability of the proposed
interpolation technique to deal with grid crossing instabilities. \textcolor{blue}{A solid elastic square block is incrementally loaded only considering body forces.} Details of the problem are sketched in figure~\ref{fig:block}
%%%%%%%%%%%%%%%%%%%%%%%%%%%%%%%%%%%%%%%%%%%%%%%%%%%%%%%%%%%%%%%%%%%%%%%%%%
\begin{figure}
%\sidecaption
  \centering
  \resizebox{0.7\hsize}{!}{
    \input{Figures-Block.tex}}
  \caption{Geometrical description of a soil block }.
  \label{fig:block}
\end{figure}
%%%%%%%%%%%%%%%%%%%%%%%%%%%%%%%%%%%%%%%%%%%%%%%%%%%%%%%%%%%%%%%%%%%%%%%%%%
This test was previously proposed by Andersen (2009)\cite{thesis_Andersen_2009}. The
elastic parameters considered for this test are: 
\begin{itemize} 
\item  Initial density : $6\cdot 10^3\ kg/m^3$
\item  Poisson ratio : $0$
\item  Elastic modulus : $5\ MPa$
\end{itemize}
The gravity force is applied as an external force. Using a total time period T of 20 seconds to apply the gravity, it is increased from 0 to 9.81$m/s$
with a sinusoidal function until T/2 seconds and then maintained constant until T
in order to reach the equilibrium state:
\begin{equation}
  \label{eq:gravity-load-block}
 \mathbf{g}(t) = \left\{
    \begin{array}{ll}
      0.5 \mathbf{g} (\sin(\frac{2t \pi}{T} - \frac{\pi}{2})+1)  & \mbox{if } t \leq T/2 \\
      \mathbf{g} & \mbox{if } t > T/2
    \end{array}
  \right.
\end{equation}
In order to get a stable solution, time step was conducted by a
Courant number of 0.1. \textcolor{blue}{The explicit
predictor-corrector scheme proposed in previous sections is employed.} For the initial spatial discretisation four particles per cell
($\Delta x = 2\ m$) were adopted. The initial layout of particles inside of the
cell changes according to the approximation technique adopted. For the
bi-linear shape functions and the \acrshort{lme} approximants, the initial
position corresponds to the location of the gauss-points in a standard
quadratic finite element. \textcolor{blue}{For the \acrshort{ugimp} shape function the initial
position of each particle is located in the center of each voxel as in the initial situation voxel domains should not overlap.}

Figure~\ref{fig:Block-LME3} shows the evolution of the vertical stress
during the loading process. The result is physically realistic as
the stress increases linearly from the top to the bottom of the specimen,
and the value of the vertical stress in a material point located in
the bottom of the specimen oscillates around $5.2 MPa$, which is
the analytic value given by $\sigma_{yy} = \rho g h_y$.
%%%%%%%%%%%%%%%%%%%%%%%%%%%%%%%%%%%%%%%%%%%%%%%%%%%%%%%%%%%%%%%%%%%%%%%%%%
\begin{figure*}
  \centering
  \subfigure[t = 0 seconds.]{
    \includegraphics[width=0.3\textwidth]{Figures-Block-LME3-PCE-a-t0}
  }
  \subfigure[t = 5 seconds.]{
    \includegraphics[width=0.3\textwidth]{Figures-Block-LME3-PCE-b-t025}
  }
  \subfigure[t = 20 seconds]{
    \includegraphics[width=0.3\textwidth]{Figures-Block-LME3-PCE-c-t1}
  }
  \caption{Vertical normal stress and position of material points
    during the loading process for a soft soil ($E = 5\ MPa$, $\rho_0
    = 6\cdot 10^3\ kg/m^3$). Numerical parameters considered for the
    simulation are : Local \textit{max-ent} shape function $\widehat{\gamma} =3$
    and explicit PC scheme with CFL 0.1.}
  \label{fig:Block-LME3}
\end{figure*}
%%%%%%%%%%%%%%%%%%%%%%%%%%%%%%%%%%%%%%%%%%%%%%%%%%%%%%%%%%%%%%%%%%%%%%%%%%

Figure~\ref{fig:vertical-displacement-block} shows the vertical
displacement evolution of a point in the free surface of the
block. \textcolor{blue}{No results can be appreciated for the \acrshort{mpm} simulation with a bi-linear interpolation technique (Q4) as a clear unstable behaviour is observed within the first time steps.The \acrshort{lme} simulation was performed using two kinds of shape functions, one with a low value of the dimensionless parameter, $\widehat{\gamma} = 0.8$, and other with a larger value of it, $\widehat{\gamma} = 3.0$. Both simulations behave with minimal oscillations and no appreciable differences are observed between them. The \acrshort{ugimp} simulation is more stable than the one performed with the \acrshort{mpm}-Q4 and behaves like the \acrshort{lme} simulations. Finally, these results are compared against two contrasted techniques. The \acrshort{fem} solution has been done with a regular mesh of linear quadrilaterals, meanwhile the \acrshort{otm} solution has been done with \acrshort{lme} approximation. The final displacement in both solutions oscillates around a value of 0.59 m.} 
%%%%%%%%%%%%%%%%%%%%%%%%%%%%%%%%%%%%%%%%%%%%%%%%%%%%%%%%%%%%%%%%%%%%%%%%%%
\begin{figure}
%\sidecaption
  \centering
  \resizebox{\hsize}{!}{
    \includegraphics[width=\textwidth]{./Figures-Block-CFL-01-Comparative}
  }
  \caption{Comparative of the vertical displacement evolution in a
    point located in the free surface employing different
    interpolation schemes and numerical techniques.} 
  \label{fig:vertical-displacement-block}
\end{figure}
%%%%%%%%%%%%%%%%%%%%%%%%%%%%%%%%%%%%%%%%%%%%%%%%%%%%%%%%%%%%%%%%%%%%%%%%%%

\subsection{2D elastic wave propagation}
\label{sec:Velocity-waves-elastic-domain}

\textcolor{blue}{This benchmark has been conveniently adapted from the one proposed by  Hammerquist \& Nairn (2017)~\cite{HAMMERQUIST2017724}, it is intended to validate the ability of the discussed time integration schemes in order to reproduce the wave propagation in 2D. The impact of an elastic square plate against a vertical wall is assessed. The square plate dimensions are 50 x 50 m and their elastic parameters are:
\begin{itemize} 
\item  Initial density : $20\ kg/m^3$
\item  Poisson ratio : $0.3$
\item  Elastic modulus : $0.1\ MPa$
\end{itemize}
}
\textcolor{blue}{The background set of nodes is composed by a cartesian grid with an uniform spatial discretisation of 1 m. Four material points per cell are considered. The Courant number is fixed to 0.1 during the whole simulation and $\text{LME}_{3.0}$ shape functions are employed. A horizontal velocity of the square plate of $\text{V}_0 = 1.0 m/s$ is considered until the the plate impacts against the obstacle generating stress waves. The obstacle is modelled as a set of nodes with prescribed velocity equal to zero. Both \acrshort{npc} and \acrshort{fe} schemes are compared. Spatial distribution of the velocity field inside the square plate obtained with both schemes at different times are given in figure ~\ref{fig:Magnitude_velocity_impact_square}}. 
%%%%%%%%%%%%%%%%
\begin{figure}
%\sidecaption
  \centering
  \resizebox{\hsize}{!}{
\includegraphics[scale=1]{Figures-Velocity-FE-vs-NPC-square.png} 
  }
  \caption{Spatial distribution of the magnitude of the velocity field in an elastic square plate impacted on the right edge. The top row depicts the \acrshort{npc} results meanwhile the bottom row represents the \acrshort{fe} ones at different instants.}
  \label{fig:Magnitude_velocity_impact_square}
\end{figure}
%%%%%%%%%%%%%%%% 
\textcolor{blue}{At early times, \acrshort{npc} and \acrshort{fe} behave in a similar manner. However, at later times, the \acrshort{fe} scheme introduces significant numerical noise compared with the \acrshort{npc} one. } 

\textcolor{blue}{Finally, an unstructured background mesh is proposed in order to validate the performance of the \acrshort{lme} shape function against the \acrshort{ugimp} one.}
%%%%%%%%%%%%%%%%
\begin{figure}
%\sidecaption
  \centering
  \resizebox{\hsize}{!}{
\includegraphics[scale=0.8]{Figures-Velocity-structured-unstructured-lme-ugimp.png}
  }
  \caption{Sensitivity analysis of \acrshort{ugimp} and \acrshort{lme} schemes to the randomness in the layout of the background set of nodes. On the left column, sketches of the two employed meshes are depicted. On the right side, spatial distribution of the velocity magnitudes are shown for each case at second 2.66. All the simulation where carried out with the proposed \acrshort{npc} scheme.}
  \label{fig:Magnitude_velocity_impact_square_structured_unestructured}
\end{figure}
%%%%%%%%%%%%%%%%
\textcolor{blue}{ As it is observed in figure~\ref{fig:Magnitude_velocity_impact_square_structured_unestructured}, the \acrshort{lme} approximation preserves the solution even though the mesh is not structured. Meanwhile, the \acrshort{ugimp} is not able to deal with the unstructured mesh as expected. %It goes without saying that there is no point in using unstructured meshes in this type of situation. 
%Despite this, 
Thus, the \acrshort{lme} approach adopted in the present work shows a remarkable robustness dealing with unstructured meshes. From this last comparison it can be deduced that \acrshort{lme}  might be an adequate choice approximation technique when dealing with complex domains where a structured mesh is usually not possible .}

\section{Conclusions}
\label{sec:conclusions}
We have proposed in this paper an enhanced methodology which improves \acrfull{mpm} behaviour in fast dynamic problems.\textcolor{blue}{The proposed improvements implemented in the present work are twofold: First, the \acrfull{npc} time integration algorithm to update particles' information, and second, the \acrfull{lme} approximation scheme to define the shape functions. The performance of both improvements has been assessed with three different benchmarks: Dyka's bar, a solid block under gravity loading and 2D elastic waves propagation.}

\textcolor{blue}{The main achievements of the proposed time discretisation have been observed within Dyka's bar \ref{sec:dyka-bar}, where lower error values are obtained compared with the \acrfull{fe} scheme ones. Furthermore, the \acrshort{npc} approach has shown a more stable performance compared with the \acrshort{fe} scheme, being able to dump out spurious numerical oscillations where the \acrshort{fe} scheme could not}. As per the promising results obtained with the \acrshort{npc} algorithm, the procedure employed in the present work opens the possibility to revisit a huge variety of time integration schemes developed originally for \acrshort{fem}, which can be implemented within a \acrshort{mpm} framework with few modifications. In addition to the implementation of enhanced time integration schemes, further research can be done in the analysis of the good performance of the algorithm within non-linear models, both material and geometric nonlinearity.

Regarding the spatial discretisation the \acrfull{lme} approximation scheme has been validated as a robust and versatile tool in the \acrshort{mpm} framework. It also comes up as a promising alternative to other approximation techniques, developed within the \acrshort{mpm} framework, in order to overcome grid crossing limitations and to avoid the constriction of the \acrshort{ugimp} of a regular mesh or a high density of particles per cell \textcolor{blue}{as it was observed in section \ref{sec:Velocity-waves-elastic-domain}}. Several research lines of the \acrshort{lme} scheme dive into the improvement of the methodology, focusing in the optimization of the calculation of the shape function and the possibility of adapting the parameter $\beta(\vec{x})$ through the \textcolor{red}{deformation gradient} in order to align the shape functions in the principal strain direction to select the most suitable set of nodes adapted to the strain. \textcolor{blue}{Another possibility could be to adapt}  the value of $\beta$ to solve the equations \acrshort{fem}-like \textcolor{red}{or} meshfree-like depending  \textcolor{red}{on the behaviour of the region}. Together with the implementation of non-linear behaviours, previously mentioned, challenging scenarios could be reproduced, being the main purpose of the research line the simultaneous simulation of both initialization and propagation stages of fast landslides.

% 
\section*{Conflict of interest}
%
The authors declare that they have no conflict of interest.

% 
\section*{Acknowledgements}
%
The financial support to develop this research from the Ministerio de Ciencia e Innovaci\'on, under Grant No. BIA-2016-76253 is greatly appreciated. The first and the second authors also acknowledge the fellowship Fundaci\'on Agust\'in de Betancourt and Juan de la Cierva (FJCI-2017–31544) respectively.


% \printglossary[type=\acronymtype]
\printglossaries

\appendix

\section{The analytical solution of the 1D Dyka benchmark}
\label{app:analytical_sol}

For the derivation of this analytical solution, a 1D elastic bar is
consider. Henceforth for convenience the governing equations will be written in terms of stress and velocity. The balance of linear momentum,
\begin{equation}
  \label{eq:1D-balance-linear-momentum}
  \rho\ \Deriv{v}{t} = \Deriv{\sigma}{x},
\end{equation}
Secondly the constitutive equation is the well known linear
elastic one,
\begin{equation}
  \label{eq:1D-constitutive-equation}
  \Deriv{\sigma}{t} = E \Deriv{\varepsilon}{t},
\end{equation}
where $E$ is the elastic modulus. And finally the compatibility
equation,
\begin{equation}
  \label{eq:CompatibilityEquation_e}
  \Deriv{\varepsilon}{t} = \Deriv{v}{x}.
\end{equation}
Next for simplicity, we will introduce
\eqref{eq:CompatibilityEquation_e} in
\eqref{eq:1D-constitutive-equation}, it yield to,
\begin{align}
  \label{eq:1D-balance-linear-momentum-II}
  \Deriv{v}{t} &= \frac{1}{\rho}\ \Deriv{\sigma}{x}, \\
  \label{eq:1D-constitutive-equation-II}
  \Deriv{\sigma}{t} &= E\ \Deriv{v}{x}.
\end{align}
Introducing \eqref{eq:1D-constitutive-equation-II} in
\eqref{eq:1D-balance-linear-momentum-II} and expressing the remaining equation in terms of the displacement, results the wave equation for linear elastic materials,
\begin{equation}
  \label{eq:1D-wave-elastic}
  \Deriv[2]{u}{t} = \frac{E}{\rho}\ \Deriv[2]{u}{x} = c^2\ \Deriv[2]{u}{x}
\end{equation}
where $c = \sqrt{\frac{E}{\rho}}$ is the material celerity. Alternative, rearranging both equations \eqref{eq:1D-balance-linear-momentum-II} and
\eqref{eq:1D-constitutive-equation-II} it is possible to join them in a single system of equations as,
\begin{equation}
  \label{eq:System-stress-velocity}
  \Deriv{}{t} \left[
    \begin{array}{c}
      \sigma \\
      v
    \end{array}
  \right] + \left[
    \begin{array}{cc}
      0 & - E \\
      - 1/\rho & 0 
    \end{array} \right] \left[
    \begin{array}{c}
      \Deriv{\sigma}{x} \\
      \Deriv{v}{x}
    \end{array}
  \right] = \Vector{0}.
\end{equation}
Or in a more compact format,
\begin{equation}
  \label{eq:System-stress-velocity-II}
  \Deriv{\Vector{\phi}}{t} + \Matrix{A}\Deriv{\Vector{\phi}}{x} = \Vector{0}.
\end{equation}
In \eqref{eq:System-stress-velocity-II} stress and velocity are joined in to a single structure $\Vector{\phi}$ and $\Matrix{A}$ in coupling matrix between both equations,
\begin{equation*}
  \Vector{\phi} = \left[
    \begin{array}{c}
      \sigma \\
      v
    \end{array}
  \right],\quad 
  \Matrix{A} =  \left[
    \begin{array}{cc}
      0 & - E\\
      - 1/\rho & 0 
    \end{array} \right].
\end{equation*}
Despite of this manipulation, the nature
of \label{eq:eq:System-stress-velocity-II} is still hyperbolic. A
proof of this can be easily obtained computing the zeros of the
hypersurface defined by \eqref{eq:1D-wave-elastic}. And later the
eigenvalues of $\Matrix{A}$ in \eqref{eq:System-stress-velocity-II}. In both cases, eigenvalues are real and distinct ($\lambda = \pm \sqrt{\frac{E}{\rho}}$),
therefore the system is called strictly hyperbolic. Assuming that  $\Matrix{A}$ has $n$ different eigenvalues $\{ \lambda_1, \ldots, \lambda_i, \ldots
\lambda_n \}$ and $n$ eigenvectors $\{ \vec{x}^1, \ldots,
\vec{x}^i, \ldots \vec{x}^n \}$ satisfying that $\tens{A} \vec{x} =
\lambda \vec{x} $. Now we introduce the matrix $\Matrix{P}$ whose columns are the $n$ eigenvalues $\Vector{x}$
\begin{equation}
  \label{eq:P-matrix}
\Matrix{P} = \{ \vec{x}^1, \vec{x}^2, \vec{x}^3, \ldots \vec{x}^n \}.
\end{equation}
Diagonalizing $\Matrix{A}$ using $\Matrix{P}$ yields,
\begin{equation}
  \label{eq:Lambda-matrix}
  \Lambda = \Matrix{P}^{-1} \Matrix{A}\ \Matrix{P},
\end{equation}
where $ \Lambda_{ii} = \lambda_i$. Now, lets define a vector $\Vector{\Re}$ as
\begin{equation}
  \label{eq:Riemann-definition}
  \Vector{\phi} = \Matrix{P}\ \Vector{\Re}.
\end{equation}
Expanding the above expression with the chain rule and passing the
matrix $\Matrix{P}$ to left hand side of the equality we get,
\begin{equation}
  \label{eq:Riemann-II}
  d \vec{\Vector{\Re}} = \Deriv{\Vector{\Re}}{t}dt + \Deriv{\Vector{\Re}}{x}dx =
  \tens{P}^{-1}\left(\Deriv{\phi}{t}dt + \Deriv{\phi}{x}dx \right)
\end{equation}
and setting the terms we get,
\begin{equation}
  \label{eq:Riemann-III}
  \Deriv{\Vector{\Re}}{t} = \Matrix{P}^{-1}\Deriv{\Vector{\phi}}{t},\quad 
  \Deriv{\Vector{\Re}}{x} = \Matrix{P}^{-1}\Deriv{\Vector{\phi}}{x}
\end{equation}
Next, if we multiply \eqref{eq:System-stress-velocity-II} by
$\Matrix{P}^{-1}$ we get:
\begin{equation}
  \label{eq:System-stress-velocity-III}
  \Matrix{P}^{-1}\Deriv{\Vector{\phi}}{t} + \left(\Matrix{P}^{-1}\Matrix{A}\Matrix{P}
  \right)\Matrix{P}^{-1} \Deriv{\Vector{\phi}}{x} = \Vector{0}
\end{equation}
finally introducing the expressions \eqref{eq:Riemann-III} we reach to
\begin{equation}
  \label{eq:System-stress-velocity-IV}
  \Deriv{\Vector{\Re}}{t} + \varLambda \Deriv{\Vector{\Re}}{x} = \Vector{0}  
\end{equation}
which consists of $n$ uncoupled equations as $\varLambda$ is
diagonal matrix as we can see in \eqref{eq:Lambda-matrix}. Each of
this equations are 1D scalar convective transport equations, with
solutions of the form:
\begin{equation}
  \label{eq:SystemEquations_sigma_v_VI}
  \Re^{(i)} = F^{(i)} \left(x - \lambda^{(i)} t \right)
\end{equation}
This uncoupled system, has a set of $n$ characteristics.
These magnitudes $\Re_i$ which propagate along characteristics are
known as \textit{Riemann invariants} of the problem. For the closure
of the problem it is required ``n'' initial conditions of the form
$\Re_i (x,t=0) = h_i(x)$, and ``n'' boundary
conditions. Particularizing the previous equations for the 1D elastic
bar described in \cite{Dyka1995}, $\Matrix{P}$ can computed as,
\begin{equation*}
    \Matrix{P} =  \left[
    \begin{array}{cc}
      -\sqrt{E\rho} & \sqrt{E\rho}\\
       1 & 1 
    \end{array} \right]
\end{equation*}
With the value of the inverse matrix $\Matrix{P}^{-1}$ in the Riemann
definition \eqref{eq:Riemann-definition}, a set of equations arise,
\begin{align}
  \label{eq:Riemann-I-1D-elastic-bar}
  &\Re^{I} = \frac{1}{2\sqrt{\rho E}}\left(-\sigma + v\ \sqrt{\rho E}
    \right)\\
  \label{eq:Riemann-II-1D-elastic-bar}
  &\Re^{II} = \frac{1}{2\sqrt{\rho E}}\left(\sigma + v\ \sqrt{\rho E} \right)
\end{align}
From \eqref{eq:Riemann-I-1D-elastic-bar} and
\eqref{eq:Riemann-II-1D-elastic-bar} the values of the
stress and the velocity can be computed in the following way,
\begin{equation}
  \label{eq:Riemann-stress-velocity}
  v = \Re^{I} + \Re^{II} \quad , \quad \sigma = \sqrt{E \rho}\left(\Re^{II} - \Re^{I} \right)
\end{equation}
The boundary conditions are in both cases of radiation as there is not
wave in-going from the exterior. So for the right side the conditions are,
\begin{equation*}
  \Re^{II} = 0 \quad and \quad v_{x=L} = 0 \quad\Rightarrow \quad \sigma_{x=L} = -2\sqrt{\rho E}\ \Re^{I}
\end{equation*} 

And in the left side,
\begin{equation*}
  \Re^{I} = 0 \quad and \quad \sigma_{x=0} = 0  \quad \Rightarrow
  \quad v_{x=0} = 2\Re^{II}
\end{equation*}
By imposing this boundary conditions, the problem is fully defined as in \cite{Dyka1995}.

%%%%%%%%%%%%%%%%%%%%%%%%%%%%%%
% name your BibTeX data base
\bibliography{Biblio}
\end{document}
%%%%%%%%%%%%%%%%%%%%%%%%%%%%%%


%%% Local Variables:
%%% mode: latex
%%% TeX-master: t
%%% End:
