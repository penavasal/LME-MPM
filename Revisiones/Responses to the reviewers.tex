\documentclass[12pt]{article}
\usepackage{geometry} % see geometry.pdf on how to lay out the page. There's lots.
\geometry{a4paper} % or letter or a5paper or ... etc
\usepackage{graphicx}
\usepackage{cite}
\usepackage{color}
\usepackage{bm}

% \geometry{landscape} % rotated page geometry

\title{Response to the reviewers of the manuscript ``On the dynamic assessment of the Local-Maximum Entropy Material Point Method through an Explicit Predictor-Corrector Scheme"}
\author{Miguel Molinos, Pedro Navas, Manuel Pastor, and Miguel Mart\'in-Stickle}
\date{September 1, 2020}

%\date{} % delete this line to display the current date

%%% BEGIN DOCUMENT
\begin{document}

\maketitle
%\tableofcontents

We  are grateful to the reviewers for taking their time  to review our work. Their comments have helped us to improve the paper. Changes to the original manuscript are given in  color (\textcolor{red}{red} for corrections, \textcolor{blue}{blue} for modified texts)  in the revised version.  A detailed response to both reviewers is given below.

\section*{Response to Reviewer \#1}
{\it
The authors have presented a promissing approach for MPM to surpass the shock wave propagation in the traditional MPM. The paper was very well written and the results suggested that the proposed approach works.
}
\\
\\
The reviewer's positive commentaries are greatly appreciated.
\\
\begin{enumerate}
\item \textit{However, most of the applications presented in this manuscript are relatively simple and does not really demonstrate the capability of the proposed approach. I suggest the authors to extend their manuscript to predict more challenging problems and compare their results to the original MPM solutions.}\\

Authors are completely agree with the proposal. And in analogy with the work of \cite{HAMMERQUIST2017724}, where fast dynamics were also studied. The Sulsky classical example of two colliding disks has been discussed in the new version of the manuscript.

\end{enumerate}

\hspace{5mm}



\section*{Response to Reviewer \#2}


\textit {This is, on the face of it, an interesting piece of work but in its present form I don't think this is ready for CMAME. The paper is too long, rambling and has many grammatical and spelling errors. }\\

Authors really appreciate that the reviewer consider the present research as an interest piece of work. In order to improve the readability of the revised paper, a proofreading is made.\\

\textit{Technical issues for the authors follow:}
 \begin{enumerate}
\item \textit{Cell-crossing is a well known issue with the MPM but in the review no mention is made of CPDI approaches.}\\

A citation of the work of \cite{Sadeghirad_2011} with the CPDI has been added to the state of art.

\item \textit{The development is very confusing and would make more sense were it to be tied to 
the way that GIMP methods are usually stated, i.e. that the standard FE shape functions are replaced by weighting functions comprised of the conjugation of a standard shape function and a particle "characteristic" function. Line 155 demonstrates this, i.e. the use of $N_{Ip}$ before explanation.}\\

The development of the MPM approach has been revised considering the proposal of the reviewer. 

\item \textit{There is little discussion on some of the pitfalls of using max-ent, e.g. the need for a convex domain, the issue of determining Lagrange Multiplier solutions for points close to boundaries, i.e. the solutions process here at Eqns 37 and 38.}\\

Further discussion has been added. 

\item \textit{Does the use of max-ent solve the problem of imposing Dirichlet boundary conditions in the MPM?}\\

No it does not. Local \textit{max-ent} approximation technique only provides to the MPM an efficient tool to transfer information from particles to nodes and vice versa. The difficulties of imposing Dirichlet boundary conditions (DBC) in the MPM are a direct consequence of discretising the continuum with material points nor with nodes where DBC are imposed. See for instance the work of \cite{Cortis_et_al_2017_IJNME}.

\item \textit{I am sceptical of the novelty of the proposed explicit scheme as there is no mention of the recent XPIC(m) approach of Hammerquist and Nairn or similar approaches used in the Computational Graphics community, e.g. Stomakhin et al. (2013) or the more recent Polypic.}\\

These publications will be added to the state of art of the manuscript. In the humble opinion of the authors, the XPIC approach consists in the stabilisation of the solution through a novel transference procedure. Meanwhile, the algorithm introduced in this research proposes the stabilisation within the adaptation of a FEM time integration scheme for the MPM framework. Hence, both techniques can be employed simultaneously.

\end{enumerate}

 \textit{The numerical examples are not impressive when the message of the paper is that this is good for fast dynamics problems. Compare to papers such as Ma et al. (2009) IJ Impact Engineering}\\

We agree with the reviewer that some of the numerical example does not enough highlights the capabilities of the MPM Newmark Predictor-Corrector scheme. Although the results presented in \cite{MA2009272} are impressive, the proposed examples are out of the scope of the present research. Therefore, in analogy with the work of \cite{HAMMERQUIST2017724}, previously cited by the reviewer, the Sulsky classical example of two colliding disks has been discussed. 

\bibliographystyle{unsrt}
\begin{thebibliography}{10}

\bibitem{Cortis_et_al_2017_IJNME}
Cortis, Michael and Augarde, Charles and Coombs, William and Robinson, Scott and Brown, Michael and Brennan, Andrew.
\newblock {Imposition of essential boundary conditions in the material point method}.
\newblock {International Journal for Numerical Methods in Engineering}, 113, 2017.

\bibitem{Sadeghirad_2011}
Sadeghirad, A. and Brannon, R. M. and Burghardt, J.
\newblock {A convected particle domain interpolation technique to extend applicability of the material point method for problems involving massive deformations}.
\newblock {International Journal for Numerical Methods in Engineering}, 86(12) : 1435-1456, 2011.


\bibitem{HAMMERQUIST2017724}
Chad C. Hammerquist and John A. Nairn.
\newblock {A new method for material point method particle updates that reduces noise and enhances stability}.
\newblock {\em Computer Methods in Applied Mechanics and Engineering},  318:724--738, 2017.


\bibitem{MA2009272}
S. Ma and X. Zhang and X.M. Qiu.
\newblock{Comparison study of MPM and SPH in modeling hypervelocity impact problems}.
\newblock{International Journal of Impact Engineering}, 36(2):272--282, 2009.

\end{thebibliography}

\end{document}


%Del revisor 2:
%1- Meter paja de eso y fuera
%2- Yo haria referencia al paper de Wobbes, como lo nombren ellos, y decir que haces referencia a un paper contrastado....
%3- Hablar de eso, que hay pequeña discusion, pues se discute mas... 
%4- No, no lo soluciona, pero en OTM lo hace. Habría que conseguir una conjunción de lo que hace OTM con lo que hacen otros metodos en MPM para imponer boundary conditions....
%5- Leerse que hacen esos pollos, si esta bien y crees que puedes mter algun ejemplo, se puede intentar y asi tb se contenta al revisor 1
