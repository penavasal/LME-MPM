\documentclass[12pt]{article}
\usepackage{geometry} % see geometry.pdf on how to lay out the page. There's lots.
\geometry{a4paper} % or letter or a5paper or ... etc
\usepackage{graphicx}
\usepackage{cite}
\usepackage{color}
\usepackage{bm}
\usepackage{amsmath}

% \geometry{landscape} % rotated page geometry

\title{Response to the reviewers of the manuscript ``On the dynamic assessment of the Local-Maximum Entropy Material Point Method through an Explicit Predictor-Corrector Scheme"}
\author{Miguel Molinos, Pedro Navas, Manuel Pastor\\
and Miguel Mart\'in-Stickle}
\date{September 1, 2020}

%\date{} % delete this line to display the current date

%%% BEGIN DOCUMENT
\begin{document}

\maketitle
%\tableofcontents

We  are grateful to the reviewers for taking their time  to review our work. Their comments have helped us to improve the paper. Changes to the original manuscript are given in  color \textcolor{red}{red} for corrections and  \textcolor{blue}{blue} for modified text in the revised version.  A detailed response to both reviewers is given below.

\section*{Response to Reviewer \#1}
{\it
The authors have presented a promissing approach for MPM to surpass the shock wave propagation in the traditional MPM. The paper was very well written and the results suggested that the proposed approach works.
}
\\
\\
The reviewer's positive commentaries are greatly appreciated.
\\
\textit{However, most of the applications presented in this manuscript are relatively simple and does not really demonstrate the capability of the proposed approach. I suggest the authors to extend their manuscript to predict more challenging problems and compare their results to the original MPM solutions.}\\

Authors agree with the proposal of the reviewer. In the first version, aiming to make a readable manuscript, several examples were not shown. In this revised version, in order to clarify several questions of both reviewers, we have introduced an additional case of study which consists in the simulation of a side impact in an elastic square. This test has been conveniently adapted from the work of Hammerquist \& Nairn \cite{HAMMERQUIST2017724}, where fast dynamics were also studied.

\hspace{5mm}



\section*{Response to Reviewer \#2}

\textit {This is, on the face of it, an interesting piece of work but in its present form I don't think this is ready for CMAME. The paper is too long, rambling and has many grammatical and spelling errors. }\\

Authors really appreciate that the reviewer consider the present research as an interest piece of work. In order to improve the readability of the revised paper, a proofreading is made.\\

\textit{Technical issues for the authors follow:}
 \begin{enumerate}
\item \textit{Cell-crossing is a well known issue with the MPM but in the review no mention is made of CPDI approaches.}\\

A citation of the work of \cite{Sadeghirad_2011} with the CPDI has been added to the state of art.

\item \textit{The development is very confusing and would make more sense were it to be tied to 
the way that GIMP methods are usually stated, i.e. that the standard FE shape functions are replaced by weighting functions comprised of the conjugation of a standard shape function and a particle "characteristic" function. Line 155 demonstrates this, i.e. the use of $N_{Ip}$ before explanation.}\\

Unfortunately, the \textit{max-ent} approximation can not be expressed as the conjugation of a standard shape function and a particle "characteristic" function. However, in order to improve the readability of the MPM development, the approach has been revised with a detailed explanation.

\item \textit{There is little discussion on some of the pitfalls of using max-ent, e.g. the need for a convex domain, the issue of determining Lagrange Multiplier solutions for points close to boundaries, i.e. the solutions process here at Eqns 37 and 38.}\\

Of course, the employment of the \textit{max-ent} approximants is not free of difficulties. Some requirements such as the need for a convex domain, as well as those derived from its calculation (the determination of the Lagrange Multipliers, specially at the boundaries, or the obtainment of the minimizer of the logarithmic function), make the usage of it a challenging tool. However, the results depicted in this manuscript will strengthen the motivation of its employment in order to mitigate typical MPM problems such as cell-crossing, stress instabilities and the use of unstructured mesh.  

Concerning the first issue mentioned by the reviewer, the special case of non-convex domains was discussed by Arroyo \& Ortiz~\cite{Arroyo2006}. Some of the solutions proposed by the aforementioned authors are : the possibility of replacing the Euclidean distance $\lVert  x - x_a  \rVert$ in the definition of the shape functions by the length of the shortest path contained within the domain connecting $x$ and $x_a$. Or the decomposition of the non-convex domain into convex sub-domains.  This topic has also been extensively studied in the context of MLS-based meshfree methods, for instance visibility, diffraction, and constrained path criteria. These methods are directly applicable to local \textit{max-ent} approximation. 


\item \textit{Does the use of max-ent solve the problem of imposing Dirichlet boundary conditions in the MPM?}\\

Unfortunately, local \textit{max-ent} approximation technique only provides an efficient tool to transfer information from particles to nodes and vice versa to the MPM. Hence, the difficulties of imposing Dirichlet boundary conditions (DBC) in the MPM persists since they are a direct consequence of discretising the continuum with material points instead of nodes, where traditionally DBC are imposed. However, any available technique employed to overcome this situation in the standard MPM can be translated to the \textit{max-ent} one thanks to the Kronecker-Delta property inherent to it \cite{Arroyo2006}. See for instance the work of Cortis {\it et al.} (2018) \cite{Cortis_et_al_2017_IJNME} as example of how DBC can be imposed in the MPM.

\item \textit{I am sceptical of the novelty of the proposed explicit scheme as there is no mention of the recent XPIC(m) approach of Hammerquist and Nairn or similar approaches used in the Computational Graphics community, e.g. Stomakhin et al. (2013) or the more recent Polypic.}\\

These interesting publications will be added to the state of art of the revised manuscript.In the humble opinion of the authors, the XPIC approach is dedicated to mitigate the null space error produced during the transfer of information between particles and nodes. Therefore, it is related to the spatial discretisation in stead of temporal. Meanwhile, the algorithm introduced in this research proposes the stabilisation within the adaptation of the predictor-corrector algorithm, which has been used traditionally in FEM to deal with fast dynamics. To the author\textquotesingle s knowledge, this is the first attempt to adapt the  traditional predictor-corrector for the MPM framework. 

\end{enumerate}

 \textit{The numerical examples are not impressive when the message of the paper is that this is good for fast dynamics problems. Compare to papers such as Ma et al. (2009) IJ Impact Engineering}\\

Contrary to the reviewer\textquotesingle s opinion, the provided results are capable to highlight the strong capabilities of the MPM Newmark Predictor-Corrector scheme (NPC). It is important to remark the extension of this research: in order to isolate the strength of the proposed methodology, covering spatial and time discretization, only linear elastic problems are assessed. 

Although the results presented in Ma {\it et al.} \cite{MA2009272} are impressive, the proposed examples are out of the scope of the present research. In order to reinforce the benefits of the proposed methodology, we propose a 2D wave propagation problem has been simulated. It is analogous to the work of Hammerquist \& Nairn \cite{HAMMERQUIST2017724} (previously cited by the reviewer). This new test highlights the capabilities of the NPC to reproduce complex wave patterns under challenging loads such as impacts. Which is more, it is able to reduce the presence of numerical noise compared with the FE solution. Additionally, the robustness of the local \textit{max-ent} approximation is exposed dealing with unstructured mesh. 

\bibliographystyle{unsrt}
\begin{thebibliography}{10}

\bibitem{Cortis_et_al_2017_IJNME}
Cortis, Michael and Augarde, Charles and Coombs, William and Robinson, Scott and Brown, Michael and Brennan, Andrew.
\newblock {Imposition of essential boundary conditions in the material point method}.
\newblock {International Journal for Numerical Methods in Engineering}, 113, 2017.

\bibitem{Sadeghirad_2011}
Sadeghirad, A. and Brannon, R. M. and Burghardt, J.
\newblock {A convected particle domain interpolation technique to extend applicability of the material point method for problems involving massive deformations}.
\newblock {International Journal for Numerical Methods in Engineering}, 86(12) : 1435-1456, 2011.


\bibitem{HAMMERQUIST2017724}
Chad C. Hammerquist and John A. Nairn.
\newblock {A new method for material point method particle updates that reduces noise and enhances stability}.
\newblock {\em Computer Methods in Applied Mechanics and Engineering},  318:724--738, 2017.


\bibitem{MA2009272}
S. Ma and X. Zhang and X.M. Qiu.
\newblock{Comparison study of MPM and SPH in modeling hypervelocity impact problems}.
\newblock{International Journal of Impact Engineering}, 36(2):272--282, 2009.

\bibitem{Arroyo2006}
Arroyo, M. and Ortiz, M..
\newblock{Local maximum-entropy approximation schemes: A seamless bridge between finite elements and meshfree methods}.
\newblock{International Journal for Numerical Methods in Engineering},  2006.

\end{thebibliography}

\end{document}

